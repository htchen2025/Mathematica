\documentclass[a4paper, 12pt, UTF8, heading=true, scheme=chinese]{ctexart}

\usepackage[a4paper, left=2cm, right=2cm, top=2cm, bottom=2cm]{geometry}
\usepackage{amsmath,amssymb,bm,graphicx,xcolor,tikz,array,booktabs,multicol,multirow,titlesec,hyperref,biblatex,algorithm2e,listings,mathtools}

%\titleformat{\section}
%  {\normalfont\Large\bfseries\raggedright}
%  {}
%  {0pt}
%  {}

\newcommand{\R}{\mathbb{R}}
\newcommand{\Z}{\mathbb{Z}}
\newcommand{\N}{\mathbb{N}}
\newcommand{\Q}{\mathbb{Q}}
\newcommand{\C}{\mathbb{C}}
\newcommand{\st}{\text{s.t.}}

\newenvironment{solution}{\par\noindent\textbf{解:}\par}{\hfill$\square$\par}
\newenvironment{proof}{\par\noindent\textbf{证明:}}{\hfill$\square$\par}
\newenvironment{problem}[1][]{\par\noindent\textbf{问题 }\textmd{#1}\par}{}

\linespread{1.5}

\title{\textbf{群 与 Galois 理论 \\ 练习}}
\author{陈宏泰 \\ 2024011131 \\ 清华大学数学科学系 \\ \texttt{cht24@mails.tsinghua.edu.cn}}
\date{\today}

\begin{document}

\maketitle

\tableofcontents
\newpage

\section{群论}

\begin{problem}[1.]
    对于群 $G$,$a,b\in G$,假如有 $a^5=e$,$a^3b=ba^3$,求证 $ab=ba$。
\end{problem}

\begin{proof}
    从 $ab = aeb$ 开始,利用 $a^5 = e$,我们有:
    $$
    ab = aeb = a(a^5)b = a^6b 
    $$
    再利用 $a^3b = ba^3$,可以得到:
    $$
    ab = a^6b = a^3(a^3b) = a^3(ba^3) = (a^3b)a^3 = (ba^3)a^3 = ba^6 = ba(a^5) = bea = ba
    $$
    因此 $ab = ba$。
\end{proof}

\begin{problem}[2.]
    1. 对于有限群 $G$,$H\subset G$ 是真子群,请证明
    $$
    G \neq \bigcup_{g\in G} gHg^{-1}
    $$
    如果是无限群,这个结果还正确吗?\\
    2.对于有限群$G$,其传递地作用在有$n$个元素的有限集$X$上,其中$n > 1$,请证明存在$g \in G$
    使得其在$X$上没有不动点。
\end{problem}

\begin{proof}
1. 对于有限群的情况,考虑共轭子群的并集的大小。设 $|G| = n$,$|H| = m$,则 $|gHg^{-1}| = m$。

由于 $H$ 是真子群,$m < n$。所有共轭子群的并集最多包含:
$$
\left|\bigcup_{g\in G} gHg^{-1}\right| \leq 1 + (n/m)(m - 1) < n
$$
因此 $G \neq \bigcup_{g\in G} gHg^{-1}$。

对于无限群,这个结果不一定成立。例如,考虑无限循环群 $G = \langle a \rangle$ 和子群 $H = \langle a^2 \rangle$,则:
$$
\bigcup_{g\in G} gHg^{-1} = \bigcup_{k\in\Z} a^k \langle a^2 \rangle a^{-k} = \langle a^2 \rangle \neq G
$$
但存在其他无限群使得等式成立的情况。
\end{proof}

\begin{problem}
    记$[n]={1,2,\ldots,n}$,请证明$S_6$不能传递地作用在$[7]$上。$S_7$能否传递地作用在$[8]$上呢?
\end{problem}

\begin{proof}
    
\end{proof}

\newpage

\section{环论部分}

\begin{problem}
确定环同态的同态核:
$$
\Z[x] \rightarrow \C \quad f(x) \rightarrow f(\sqrt{2} + \sqrt{3} + \sqrt{5})
$$
它是主理想吗?
\end{problem}

\begin{solution}
首先需要找到 $\alpha = \sqrt{2} + \sqrt{3} + \sqrt{5}$ 的极小多项式。

计算 $\alpha$ 的幂次:
\begin{align*}
\alpha^2 &= 2 + 3 + 5 + 2(\sqrt{6} + \sqrt{10} + \sqrt{15}) = 10 + 2(\sqrt{6} + \sqrt{10} + \sqrt{15}) \\
\alpha^4 &= (\alpha^2)^2 = \cdots
\end{align*}
通过计算可得极小多项式为 $x^4 - 10x^2 + 1$。

因此同态核为 $\ker = (x^4 - 10x^2 + 1)$,这是一个主理想。
\end{solution}

\begin{problem}
对于 $A = \R[X, Y]/(X^2 + Y^2 - 1)$,证明 $A$ 不是唯一分解整环。
\end{problem}

\begin{proof}
考虑元素 $x$ 和 $y$($X$ 和 $Y$ 在商环中的像)。则有:
$$
x^2 = 1 - y^2 = (1 - y)(1 + y)
$$
但 $x$ 是不可约元,且 $1-y$ 和 $1+y$ 也不是 $x$ 的相伴元,因此 $A$ 不是唯一分解整环。
\end{proof}

\section{Galois 理论部分}

\begin{problem}
请证明多项式 $x^4 + 3x + 3$ 在 $\Q[\sqrt[3]{2}]$ 上不可约。
\end{problem}

\begin{proof}
设 $\alpha$ 是 $x^4 + 3x + 3$ 的一个根。考虑域扩张:
$$
[\Q(\alpha, \sqrt[3]{2}) : \Q] = [\Q(\alpha, \sqrt[3]{2}) : \Q(\sqrt[3]{2})] \cdot [\Q(\sqrt[3]{2}) : \Q]
$$
由于 $[\Q(\sqrt[3]{2}) : \Q] = 3$,只需证明 $[\Q(\alpha, \sqrt[3]{2}) : \Q(\sqrt[3]{2})] = 4$。

假设 $x^4 + 3x + 3$ 在 $\Q(\sqrt[3]{2})$ 上可约,则它会有次数为 1 或 2 的因子。但通过计算模不同素数可以验证这是不可能的,因此该多项式不可约。
\end{proof}

\end{document}