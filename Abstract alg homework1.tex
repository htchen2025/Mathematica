\documentclass[a4paper, 12pt, UTF8, heading=true, scheme=chinese]{ctexart}

\usepackage[a4paper, left=2cm, right=2cm, top=2cm, bottom=2cm]{geometry}
\usepackage{amsmath,amssymb,bm,graphicx,xcolor,tikz,array,booktabs,multicol,multirow,titlesec,hyperref,biblatex,algorithm2e,listings}

\titleformat{\section}
  {\normalfont\Large\bfseries\raggedright}
  {}
  {0pt}
  {}

\newcommand{\R}{\mathbb{R}}
\newcommand{\Z}{\mathbb{Z}}
\newcommand{\N}{\mathbb{N}}
\newcommand{\Q}{\mathbb{Q}}
\newcommand{\C}{\mathbb{C}}
\newcommand{\st}{\text{s.t.}}

\newenvironment{solution}{\par\noindent\textbf{解:}\par}{\hfill$\square$\par}
\newenvironment{proof}{\par\noindent\textbf{证明:}}{\hfill$\square$\par}
\newenvironment{problem}[1][]{\par\noindent\textmd{#1}\par}{}

\linespread{1.5}

\title{\textbf{群 与 Galois 理论 \\ 作业1}}
\author{陈宏泰 \\ 清华大学数学科学系 \\ \texttt{cht24@mails.tsinghua.edu.cn}}
\date{\today} 

\begin{document}

\maketitle 

\tableofcontents
\newpage

\section{A. 乘积结构}
    \begin{problem}[A1)]
    \end{problem}
    \begin{proof}
        性质1:结合律, $\forall g_1,g_2,g_3\in G_1, h_1,h_2,h_3\in G_2$,有
        \begin{align*}
            &\ \ \ \ ((g_1,g_2)\cdot(g_2,h_2))\cdot(g_3,h_3)=
            (g_1\cdot_{_1} g_2,h_1\cdot_{_2} h_2)\cdot(g_3,h_3)\\&=
            ((g_1\cdot_{_1} g_2)\cdot_{_1} g_3,(h_1\cdot_{_2} h_2)\cdot_{_2} h_3)=
            (g_1\cdot_{_1} (g_2\cdot_{_1} g_3),h_1\cdot_{_2}(h_2\cdot_{_2} h_3))\\&=
            (g_1,h_1)\cdot(g_2\cdot_{_1} g_3,h_2\cdot_{_2} h_3)=(g_1,h_1)\cdot((g_2,h_2)\cdot(g_3,h_3)).
        \end{align*}
        
        性质2:单位元, 设$1_{_1},1_{_2}$分别为$G_1,G_2$的单位元, 则$\forall g\in G_1,h\in G_2$,有
        \begin{align*}
            (g,h)\cdot(1_{_1},1_{_2})=(g\cdot_{_1} 1_{_1},h\cdot_{_2} 1_{_2})=(g,h),\\
            (1_{_1},1_{_2})\cdot(g,h)=(1_{_1}\cdot_{_1} g,1_{_2}\cdot_{_2} h)=(g,h).
        \end{align*}

        性质3:逆元, $\forall g\in G_1,h\in G_2$,设$g^{-1},h^{-1}$分别为$g,h$的逆元, 则
        \begin{align*}
            (g,h)\cdot(g^{-1},h^{-1})=(g\cdot_{_1} g^{-1},h\cdot_{_2} h^{-1})=(1_{_1},1_{_2}),\\
            (g^{-1},h^{-1})\cdot(g,h)=(g^{-1}\cdot_{_1} g,h^{-1}\cdot_{_2} h)=(1_{_1},1_{_2}).
        \end{align*}

        综上, 在以上乘法下, $G_1\times G_2$构成一个群, 且其单位元是$(1_{_1},1_{_2})$.
    \end{proof}

    \begin{problem}[A2)]
    \end{problem}
    \begin{proof}
        $\forall (g_1,h_1),(g_2,h_2)\in G_1\times G_2$,有
        \begin{align*}
            &\ \ \ \ \pi_1(((g_1,h_1)\cdot(g_2,h_2)))=\pi((g_1\cdot_{_1} g_2,h_1\cdot_{_2} h_2)) \\&=
            g_1\cdot_{_1} g_2 = \pi_1((g_1,h_1))\cdot_{_1} \pi_1((g_2,h_2)),\\
            &\ \ \ \ \pi_2(((g_1,h_1)\cdot(g_2,h_2)))=\pi((g_1\cdot_{_1} g_2,h_1\cdot_{_2} h_2)) \\&=
            h_1 \cdot_{_2} h_2 = \pi_2((g_1,h_1))\cdot_{_2} \pi_2((g_2,h_2)).
        \end{align*}
        故$\pi_1,\pi_2$为群同态.
        而显然$\ker (\pi_1)=\{(1_{_1},h) \mid h\in G_2\},\ker (\pi_2)=\{(g,1_{_2}) \mid g \in G_2\}$.
    \end{proof}

    \begin{problem}[A3)]
    \end{problem}
    \begin{proof}
        存在性:令$G=G_1 \times G_2$, $p_1=\pi_1$, $p_2=\pi_2$,其中$\pi_1,\pi_2$为A2)中的投影映射.\\
        则对任意的群$H$以及任意的群同态$\varphi_i:H \to G_i\ (i=1,2)$存在
        $$
        \psi:H \to G,\ h \mapsto (\varphi_1(h),\varphi_2(h)),
        $$
        使得$p_i \circ \psi = \varphi_i \ (i=1,2)$. 若另外存在$\psi'$也满足上述条件, 设$\psi'(h)=(g_1,g_2)$
        则对任意$h\in H$,有$p_i(\psi'(h))=\varphi_i(h)=p_i(\psi(h))\ (i=1,2)$,所以$g_1=\varphi_1(h),g_2=\varphi_2(h)$,
        即$\psi'=\psi$. \\
        $\psi$唯一性得证.故而$G,p_1,p_2$满足题设要求,存在性得证.
        
        唯一性:由存在性已知$G_1 \times G_2,\pi_1,\pi_2$满足题设要求.
        设另外存在$G,p_1,p_2$也满足题设要求.

        先证明$G \simeq G_1 \times G_2$:
        
        考虑$H=G_1 \times G_2$, $\varphi_i=\pi_i\ (i=1,2)$,
        则存在唯一的$\psi:G_1 \times G_2 \to G$,使得$p_i \circ \psi = \pi_i\ (i=1,2)$.

        另一方面,考虑$H=G$, $\varphi_i=p_i\ (i=1,2)$,
        则存在唯一的$\theta:G \to G_1 \times G_2$,使得$\pi_i \circ \theta = p_i\ (i=1,2)$. (存在性已证)

        考虑复合映射$\theta \circ \psi: G_1 \times G_2 \to G_1 \times G_2$.有
        $$
        \pi_i \circ (\theta \circ \psi) = (\pi_i \circ \theta) \circ \psi = p_i \circ \psi = \pi_i.
        $$
        但恒等映射$\mathrm{id}_{G_1 \times G_2}$也满足$\pi_i \circ \mathrm{id}_{G_1 \times G_2}=\pi_i$,
        由$G_1 \times G_2$的泛性质, $\theta \circ \psi = id_{G_1 \times G_2}$.

        同理,考虑复合映射$\psi \circ \theta: G \to G$.有
        $$
        p_i \circ (\psi \circ \theta) = (p_i \circ \psi) \circ \theta = \pi_1 \circ \theta = p_i.
        $$
        由$G$的泛性质, $\psi \circ \theta = id_G$.

        因此, $\psi$和$\theta$是互逆的同构,即$G \simeq G_1 \times G_2$.

        再说明$p_i$在同构意义下唯一:

        设$f: G \to G_1 \times G_2$为同构映射,则$p_i=\pi_i \circ f$.则$p_i$本质上就是$G$到$G_1,G_2$的投影映射.

        综上, $G,p_1,p_2$在同构意义下唯一.且$G=G_1 \times G_2$,特别地, 我们有如下集合的同构:
        $$
        \operatorname{Hom}(H,G_1 \times G_2) \simeq \operatorname{Hom}(H,G_1) \times \operatorname{Hom}(H,G_2),\ \psi \mapsto (\pi_1 \circ \psi, \pi_2 \circ \psi).
        $$
    \end{proof}

    \begin{problem}[A4)]
    \end{problem}
    \begin{proof}
        由A3)对泛性质的证明,只需要证明
        $$
        \varphi_i: \Z / n_1 n_2 \Z \to \Z / n_i \Z, \quad \bar{k} \mapsto k\left(\bmod\  n_i\right), \quad i=1,2
        $$
        是群同态即可.先证明映射是良定义的:
        若$\bar{x}=\bar{y} $, 即$x\equiv y \pmod{n_1n_2}$,则 $n_1n_2 \mid (x - y)$。由于 $n_1 \mid n_1n_2$ 且 $n_2 \mid n_1n_2$,有:
        \[
        n_1 \mid (x - y),\ n_2 \mid (x - y)
        \]
        因此 $x\equiv y \pmod{n_1}$ 且 $x\equiv y \pmod{n_2}$,故 $\varphi$ 是良定义的.

        再证明映射保持群运算:
        $\forall \bar{a},\bar{b} \in \Z / n_1 n_2 \Z$,有$\bar{a}+\bar{b}=\overline{a+b}$,而
        $$
        \varphi_i(\bar{a}+\bar{b}) = \varphi_i(\overline{a+b}) = (a+b)(\bmod n_i) = a(\bmod n_i) + b(\bmod n_i) = \varphi_i(\bar{a}) + \varphi_i(\bar{b}).
        $$
        故$\varphi_i\ (i=1,2)$为群同态.
        
        由A3)的结论可知, 存在唯一的
        $$
        \psi: \Z / n_1 n_2 \Z \to \Z / n_1 \Z \times \Z / n_2 \Z,\ \bar{k} \mapsto (k(\bmod\ n_1),k(\bmod\ n_2)),
        $$ 
        使得$\pi_i \circ \psi = \varphi_i\ (i=1,2)$. 显然$\psi$为群同态.
        而$|\Z / n_1 n_2 \Z|=|\Z / n_1 \Z \times \Z / n_2 \Z|=n_1n_2$, 只需证明$\psi$为单射即可.
        设$\bar{a}\in \ker(\psi)$, 有
        $$
        \pi_i \circ \psi (\bar{a}) = \pi_i (0) = 0 = a(\bmod\ n_i) = \varphi_i(\bar{a}),\ i=1,2, 
        $$
        则$\bar{a}=\bar{0}$, 即$\ker(\psi)={\bar{0}}$, $\psi$为单射. 因此, $\psi$为同构映射, 即
        $$
        \Z / n_1 n_2 \Z \xrightarrow{\simeq} \Z / n_1 \Z \times \Z / n_2 \Z.
        $$
    \end{proof}

    \begin{problem}[A5)]
    \end{problem}
    \begin{proof}
        设有限的循环群$G$的阶为$n$, 上课已经证明$G\simeq \Z/n\Z$.

        设$|C_1|=n_1,|C_2|=n_2$, 则$C_1\simeq \Z/{n_1}\Z$, $C_2\simeq \Z/{n_2}\Z$.由A4)可知,
        $$
        C_1 \times C_2 \simeq \Z/{n_1}\Z \times \Z/{n_2}\Z \simeq \Z/{n_1 n_2}\Z.
        $$
        故$C_1 \times C_2$为循环群, 且其阶为$n_1 n_2$.
    \end{proof}
    \begin{problem}[A6)]
    \end{problem}
    \begin{proof}
        性质1: $(A_1\times A_2,+)$为交换群,其中,$(0_{_1},0_{_2})$为单位元.

        由A1)知$(A_1\times A_2,+)$为群,$(0_{_1},0_{_2})$为单位元. 而$\forall (a_1,a_2),(b_1,b_2)\in A_1\times A_2$, 有
        $$
        (a_1,a_2)+(b_1,b_2)=(a_1+_{_1}b_1,a_2+_{_2}b_2)=(b_1+_{_1}a_1,b_2+_{_2}a_2)=(b_1,b_2)+(a_1,a_2),
        $$
        故$(A_1\times A_2,+)$为交换群.

        性质2: 
        \begin{itemize}
            \item[$-$] 乘法满足结合律, $\forall g_1,g_2,g_3\in A_1, h_1,h_2,h_3\in A_2$,有
                \begin{align*}
                    &\ \ \ \ ((g_1,g_2)\cdot(g_2,h_2))\cdot(g_3,h_3)=
                    (g_1\cdot_{_1} g_2,h_1\cdot_{_2} h_2)\cdot(g_3,h_3)\\&=
                    ((g_1\cdot_{_1} g_2)\cdot_{_1} g_3,(h_1\cdot_{_2} h_2)\cdot_{_2} h_3)=
                    (g_1\cdot_{_1} (g_2\cdot_{_1} g_3),h_1\cdot_{_2}(h_2\cdot_{_2} h_3))\\&=
                    (g_1,h_1)\cdot(g_2\cdot_{_1} g_3,h_2\cdot_{_2} h_3)=(g_1,h_1)\cdot((g_2,h_2)\cdot(g_3,h_3)).
                \end{align*}
            \item[$-$] $(1_{_1},1_{_2})$是乘法单位元. $\forall g\in A_1,h\in A_2$,有
                \begin{align*}
                    (g,h)\cdot(1_{_1},1_{_2})=(g\cdot_{_1} 1_{_1},h\cdot_{_2} 1_{_2})=(g,h),\\
                    (1_{_1},1_{_2})\cdot(g,h)=(1_{_1}\cdot_{_1} g,1_{_2}\cdot_{_2} h)=(g,h).
                \end{align*}
        \end{itemize}

        性质3: 乘法分配律成立, $\forall (a_1,a_2),(b_1,b_2),(c_1,c_2)\in A_1\times A_2$,有
        \begin{align*}
            &\ \ \ \ ((a_1,a_2)+(b_1,b_2))\cdot(c_1,c_2)=(a_1+_{_1}b_1,a_2+_{_2}b_2)\cdot(c_1,c_2)\\&=
            ((a_1+_{_1} b_1)\cdot_{_1} c_1,(a_2+_{_2} b_2)\cdot_{_2} c_2)\\&=   
            (a_1\cdot_{_1} c_1 +_{_1} b_1\cdot_{_1} c_1,a_2\cdot_{_2} c_2 +_{_2} b_2\cdot_{_2} c_2)\\&=
            (a_1\cdot_{_1} c_1,a_2\cdot_{_2} c_2)+(b_1\cdot_{_1} c_1,b_2\cdot_{_2} c_2)\\&=
            (a_1,a_2)\cdot(c_1,c_2)+(b_1,b_2)\cdot(c_1,c_2),\\
            &\ \ \ \ (a_1,a_2)\cdot((b_1,b_2)+(c_1,c_2))=(a_1,a_2)\cdot(b_1+_{_1}c_1,b_2+_{_2}c_2)\\&=
            (a_1\cdot_{_1}(b_1+_{_1} c_1),a_2\cdot_{_2}(b_2+_{_2} c_2))\\&=
            (a_1\cdot_{_1} b_1 +_{_1} a_1\cdot_{_1} c_1,a_2\cdot_{_2} b_2 +_{_2} a_2\cdot_{_2} c_2)\\&=
            (a_1\cdot_{_1} b_1,a_2\cdot_{_2} b_2)+(a_1\cdot_{_1} c_1,a_2\cdot_{_2} c_2)\\&=
            (a_1,a_2)\cdot(b_1,b_2)+(a_1,a_2)\cdot(c_1,c_2).
        \end{align*}

        综上, $A_1 \times A_2$在以上运算下是环.

        环同态的证明: $\forall (g_1,h_1),(g_2,h_2)\in A_1\times A_2$,有
        \begin{align*}
            &\ \ \ \ \pi_1(((g_1,h_1)+(g_2,h_2)))=\pi((g_1+_{_1} g_2,h_1+_{_2} h_2)) 
            \\&=g_1+_{_1} g_2 = \pi_1((g_1,h_1))+_{_1} \pi_1((g_2,h_2)),\\
            &\ \ \ \ \pi_1(((g_1,h_1)\cdot(g_2,h_2)))=\pi((g_1\cdot_{_1} g_2,h_1\cdot_{_2} h_2)) \\&=g_1\cdot_{_1} g_2 = \pi_1((g_1,h_1))\cdot_{_1} \pi_1((g_2,h_2)),
        \end{align*}

        故$\pi_1$为环同态. 同理, $\pi_2$也为环同态.
    \end{proof}

    \begin{problem}[A7)]
    \end{problem}
    \begin{proof}
        存在性:令$A=A_1 \times A_2$, $p_1=\pi_1$, $p_2=\pi_2$,其中$\pi_1,\pi_2$为A6)中的投影映射.\\
        则对任意的环$B$以及任意的环同态$\varphi_i:B \to A_i\ (i=1,2)$存在
        $$  
        \psi:B \to A,\ b \mapsto (\varphi_1(b),\varphi_2(b)),
        $$
        使得$p_i \circ \psi = \varphi_i \ (i=1,2)$. 若另外存在$\psi'$也满足上述条件, 设$\psi'(b)=(a_1,a_2)$
        则对任意$b\in B$,有$p_i(\psi'(b))=\varphi_i(b)=p_i(\psi(b))\ (i=1,2)$,所以$a_1=\varphi_1(b),a_2=\varphi_2(b)$,
        即$\psi'=\psi$. \\
        $\psi$唯一性得证.故而$A,p_1,p_2$满足题设要求,存在性得证.

        唯一性:由存在性已知$A_1 \times A_2,\pi_1,\pi_2$满足题设要求.
        设另外存在$A,p_1,p_2$也满足题设要求.    

        先证明$A \simeq A_1 \times A_2$:

        考虑$B=A_1 \times A_2$, $\varphi_i=\pi_i\ (i=1,2)$,
        则存在唯一的$\psi:A_1 \times A_2 \to A$,使得$p_i \circ \psi = \pi_i\ (i=1,2)$.

        另一方面,考虑$B=A$, $\varphi_i=p_i\ (i=1,2)$,
        则存在唯一的$\theta:A \to A_1 \times A_2$,使得$\pi_i \circ \theta = p_i\ (i=1,2)$. (存在性已证)

        考虑复合映射$\theta \circ \psi: A_1 \times A_2 \to A_1 \times A_2$.有
        $$
        \pi_i \circ (\theta \circ \psi) = (\pi_i \circ \theta) \circ \psi = p_i \circ \psi = \pi_i.
        $$
        但恒等映射$\mathrm{id}_{A_1 \times A_2}$也满足$\pi_i \circ \mathrm{id}_{A_1 \times A_2}=\pi_i$,
        由$A_1 \times A_2$的泛性质, $\theta \circ \psi = id_{A_1 \times A_2}$.

        同理,考虑复合映射$\psi \circ \theta: A \to A$.有
        $$
        p_i \circ (\psi \circ \theta) = (p_i \circ \psi) \circ \theta = \pi_1 \circ \theta = p_i.
        $$
        由$A$的泛性质, $\psi \circ \theta = id_A$.

        因此, $\psi$和$\theta$是互逆的同构,即$A \simeq A_1 \times A_2$.

        再说明$p_i$在同构意义下唯一:

        设$f: A \to A_1 \times A_2$为同构映射,则$p_i=\pi_i \circ f$.则$p_i$本质上就是$A$到$A_1,A_2$的投影映射.

        综上, $A,p_1,p_2$在同构意义下唯一.且$A \simeq A_1 \times A_2$.
    \end{proof}

    \begin{problem}[A8)]
    \end{problem}
    \begin{proof}
        由A7)对泛性质的证明,只需要证明
        $$
        \varphi_i: \Z / n_1 n_2 \Z \to \Z / n_i \Z, \quad \bar{k} \mapsto k\left(\bmod \ n_i\right), \quad i=1,2
        $$
        是环同态即可. A4)中已经证明了$\varphi_i$是良定义的且为(加法)群同态, 只需再验证映射保持乘法和乘法单位元即可.
        $\forall \bar{a},\bar{b} \in \Z / n_1 n_2 \Z$,有$\bar{a}\cdot \bar{b}=\overline{ab}$,而
        $$
        \varphi_i(\bar{a}\cdot\bar{b}) = \varphi_i(\overline{ab}) = ab(\bmod\ n_i) = (a(\bmod\ n_i))\cdot(b(\bmod\ n_i)) = \varphi_i(\bar{a})\cdot\varphi_i(\bar{b}).
        $$
        乘法单位元$\bar{1} \in \Z / n_1 n_2 \Z$, 由中国剩余定理, 
        $$
        x\equiv 1\pmod{n_1n_2}\Leftrightarrow x\equiv 1\pmod{n_1}\quad\text{且}\quad x\equiv 1\pmod{n_2}
        $$
        故$\varphi_i(\bar{1})=1(\bmod \ n_i)$为$\Z / n_i \Z$的乘法单位元.从而$\varphi_i$为环同态. 
        
        由A7)的结论可知, 存在唯一的
        $$
        \psi: \Z / n_1 n_2 \Z \to \Z / n_1 \Z \times \Z / n_2 \Z,\ \bar{k} \mapsto (k(\bmod\ n_1),k(\bmod\ n_2)),
        $$ 
        使得$\pi_i \circ \psi = \varphi_i\ (i=1,2)$, 显然$\psi$为环同态.
        而$|\Z / n_1 n_2 \Z|=|\Z / n_1 \Z \times \Z / n_2 \Z|=n_1n_2$, 只需证明$\psi$为单射即可.
        设$\bar{a}\in \ker(\psi)$, 有
        $$
        \pi_i \circ \psi (\bar{a}) = \pi_i (0) = 0 = a(\bmod\ n_i) = \varphi_i(\bar{a}),\ i=1,2, 
        $$
        则$\bar{a}=\bar{0}$, 即$\ker(\psi)={\bar{0}}$, $\psi$为单射. 因此, $\psi$为同构映射, 即
        $$
        \Z / n_1 n_2 \Z \xrightarrow{\simeq} \Z / n_1 \Z \times \Z / n_2 \Z.
        $$
    \end{proof}

    \begin{problem}[A9)]
    \end{problem}
    \begin{proof}
        由A7)对泛性质的证明,只需要证明
        \begin{align*}
            \varphi_1: (A \times_{_\text{ring}} B)^\times \to A^\times ,\ (a,b) \mapsto a\\
            \varphi_2: (A \times_{_\text{ring}} B)^\times \to B^\times ,\ (a,b) \mapsto b
        \end{align*}
        是群同态即可, 运算是$A^\times$与$B^\times$上的乘法运算. 显然有
        $$
        (a,b)\in (A \times_{_\text{ring}} B)^\times \Leftrightarrow a\in A^\times, b\in B^\times,$$
        故映射是良定义的. $\forall (a_1,b_1),(a_2,b_2)\in \left(A \times_{_\text{ring}} B\right)^\times$,有
        \begin{align*}
            \varphi_1((a_1,b_1)\cdot(a_2,b_2))=\varphi_1((a_1\cdot a_2,b_1\cdot b_2))
            =a_1\cdot a_2=\varphi_1((a_1,b_1))\cdot \varphi_1((a_2,b_2)),\\
            \varphi_2((a_1,b_1)\cdot(a_2,b_2))=\varphi_2((a_1\cdot a_2,b_1\cdot b_2))
            =b_1\cdot b_2=\varphi_2((a_1,b_1))\cdot \varphi_2((a_2,b_2)).
        \end{align*}
        故$\varphi_1,\varphi_2$为群同态.

        由A7)的结论可知, 存在唯一的
        $$
        \psi: (A \times_{_\text{ring}} B)^\times \to A^\times \times B^\times, (a,b) \mapsto (a,b),
        $$ 
        使得$\pi_i \circ \psi = \varphi_i\ (i=1,2)$. 显然$\psi$为群同态.
        而$|(A \times_{_\text{ring}} B)^\times|=|A^\times \times B^\times|=|A^\times||B^\times|$, 只需证明$\psi$为单射即可.
        设$(a,b)\in \ker(\psi)$, 有
        \begin{align*}
            \pi_1 \circ \psi ((a,b)) = \pi_1 (1_{_A},1_{_B}) = 1_{_A} = a = \varphi_1((a,b)),\\
            \pi_2 \circ \psi ((a,b)) = \pi_2 (1_{_A},1_{_B}) = 1_{_B} = b = \varphi_2((a,b)), 
        \end{align*}
        则$(a,b)=(1_{_A},1_{_B})$, 即$\ker(\psi)={(1,1)}$, $\psi$为单射. 因此, $\psi$为同构映射, 即
        $$
        (A \times_{_\text{ring}} B)^\times \simeq A^\times \times B^\times.
        $$
    \end{proof}

\newpage 

\section{B. 域的有限乘法子群是循环群}
    \begin{problem}[B1)]
    \end{problem}
    \begin{proof}
        显然有
        \begin{align*}
            &\ \ \ \ \ x\in (\Z/n\Z)^\times \\ 
            &\Leftrightarrow \exists y\in (\Z/n\Z)^\times, \st\ xy=1\in (\Z/n\Z)^\times \\
            &\Leftrightarrow \exists y,m, \st\ xy+mn=1 \\
            &\Leftrightarrow (x,n)=1
        \end{align*}

        所以,$|(\Z/n\Z)^\times|=\phi (n)$.
    \end{proof}

    \begin{problem}[B2)]
    \end{problem}
    \begin{proof}
        由A8), A9)可知,
        $$
        (\Z/n m \Z)^\times 
        \simeq (\Z/n \Z\times_{_\text{ring}} \Z/m \Z)^\times 
        \simeq(\Z/n \Z)^\times \times_{_\text{group}} (\Z/m \Z)^\times.
        $$
        对两边取阶,有$\phi(nm)=\phi(n)\phi(m)$.

        进一步,如果$n=p_1^{\alpha_1}\cdots p_k^{\alpha_k}$,则由上式知
        $$
        \phi(n)=\phi(p_1^{\alpha_1})\cdots \phi(p_k^{\alpha_k}).
        $$
        而$\phi(p_i^{\alpha_i})=p_i^{\alpha_i}-p_i^{\alpha_i-1}=p_i^{\alpha_i}(1-\frac{1}{p_i})$.因此
        $$
        \phi(n)=n(1-\frac{1}{p_1})\cdots(1-\frac{1}{p_k}).
        $$
    \end{proof}

    \begin{problem}[B3)]
    \end{problem}
    \begin{proof}
        对任意正整数$n$, 由B1)知$|(\Z/n\Z)^\times|=\phi(n)$, 
        $\forall a\in (\Z/n\Z)^\times$, 由Lagrange定理有
        $a^{\phi(n)}=1\in (\Z/n\Z)^\times$, 即$a^{\phi(n)}\equiv 1 \pmod{n}$.
        因此, $\forall a\in \Z$且$(a,n)=1$, 有$a^{\phi(n)}\equiv 1 \pmod{n}$.

        特别地, 当$n=p$为素数时, $|(\Z/p\Z)^\times|=p-1$, 有$a^{p-1}\equiv 1 \pmod{p}$, 即Fermat小定理.

    \end{proof}

    \begin{problem}[B4)]
    \end{problem}
    \begin{proof}
        由于$d$是$n$的因子, 知道$d\mid n$, 设$m=n/d$, 于是令$C_d=\langle m \rangle$, 
        于是$C_d$是$\Z/n\Z$的子群, 且$|C_d|=d$. 
        
        下面证明$C_d$是唯一的:
        设$H$是$\Z/n\Z$的一个$d$阶子群, 则$\forall a \in H$, 有$da \equiv 0 \pmod n$(由Lagrange定理). 
        于是$n\mid da$, 由于$n=md$, 则$md \mid da$, 即$m \mid a$. 于是$H \subset \langle m \rangle = C_d$. 又$|H|=|C_d|=d$, 故$H=C_d$. 因此, $C_d$是唯一的.

        进一步, 设$H$是$\Z/n\Z$的任意子群, 设$|H|=d$, 则$d\mid n$. 由上面已经证明了存在唯一的$d$阶子群$C_d$,
        故$H=C_d$. 因此, $\Z/n\Z$的子群与$n$的因子之间存在一一对应关系, 且形如$C_d$.

    \end{proof}

    \begin{problem}[B5)]
    \end{problem}
    \begin{proof}
        对任意正整数$n$, 有
        \begin{align*}
            \sum_{d|n} \phi(d) 
            &= \sum_{d|n} \phi(\frac{n}{d}) \quad (d \text{与} \frac{n}{d} \text{都遍历} n \text{的所有因子})\\
            &= \sum_{d|n} \lvert \{1\leq k\leq \frac{n}{d}\mid (k,\frac{n}{d})=1\} \rvert \\
            &= \sum_{d|n} \lvert \{d\leq kd\leq n \mid (kd,n)=d \} \rvert \\
            &= \sum_{d|n} \lvert \{d\leq m\leq n \mid (m,n)=d \} \rvert.
        \end{align*}

        注意到最后一个等式决定了$\{1,2,\ldots,n\}$的一个划分, 与划分对应的等价关系是:
        $$
        m_1 \sim m_2 \Leftrightarrow (m_1,n)=(m_2,n).
        $$
        因此上式等于$n$, 即
        $$
        \sum_{d|n} \phi(d) = n.
        $$
    \end{proof}

    \begin{problem}[B6)]
    \end{problem}
    \begin{proof}
        对每个$d\mid n$, 定义:
        $$
        G_d=\left\{g\in G \mid \text{ord}(g)=d\right\}
        $$
        由Lagrange定理, 每个元素的阶整除$n$, 故
        $$
        G=\bigsqcup_{d\mid n} G_d \quad \text{(不交并)}
        $$ 

        现在固定$d\mid n$. 如果$G_d=\emptyset$, 则$|G_d|=0$.

        如果$G_d\neq \emptyset$, 取$g\in G_d$, 则$\text{ord}(g)=d$. 考虑循环子群$H=\langle g \rangle$, 它是$G$的$d$阶子群. 对于$a\in H$, 设$a=g^m$, 于是
        \begin{align*}
            a\text{的阶为} d 
            &\Leftrightarrow a^d=1 \text{且} a^k\neq 1, \forall 0<k<d \\
            &\Leftrightarrow g^{md}=1 \text{且} g^{mk}\neq 1, \forall 0<k<d \\
            &\Leftrightarrow d\mid md \text{且} d\nmid mk, \forall 0<k<d \\
            &\Leftrightarrow (m,d)=1.
        \end{align*}
        故而$H$中恰有$\phi(d)$个$d$阶元素. 
        又在域$K$中, $x^d-1$最多有$d$个根, 而$H=\{1,g,g^2,\ldots,g^{d-1}\}$中已经提供了$d$个互不相同的根, 故所有$d$阶元素都在$H$中.
        因此, $|G_d|=\phi(d)$当$G_d\neq \emptyset$, 否则为$0$. 因此
        $$
        n=|G|=\sum_{d\mid n} |G_d| = \sum_{d\mid n, G_d \neq \emptyset} \phi(d).
        $$
    \end{proof}

    \begin{problem}[B7)]
    \end{problem}
    \begin{proof}
        由B6)知, 
        $$
        n=\sum_{d\mid n, G_d \neq \emptyset} \phi(d).
        $$ 
        由B5)知, 
        $$
        n=\sum_{d\mid n} \phi(d).
        $$
        因此, $\sum_{d\mid n, G_d = \emptyset} \phi(d)=0$. 由于$\phi(d)>0$, 故$G_d \neq \emptyset, \forall d\mid n$.

        设$g_n\in G_n$, 则$\text{ord}(g_n)=n$. 于是$g_n$为$G$的一个生成元, 故$G$为循环群.
    \end{proof}

    \begin{problem}[B8)]
    \end{problem}
    \begin{proof}
        $\Z/p\Z$为域, $(\Z/p\Z)^\times$为$\Z/p\Z$的乘法群. 由B7)知, $\Z/p\Z$的有限子群是循环群.
    \end{proof}

    \begin{problem}[B9)]
    \end{problem}
    \begin{problem}[B10)]
    \end{problem}
    \begin{problem}[B11)]
    \end{problem}
    以上三问在初等数论课程中已被证明, 这里不再赘述.

\newpage

\section{C. 有限生成的群}
    \begin{problem}[C1)]
    \end{problem}
    \begin{proof}
        设$G$是有限生成的群, $G=\langle S \rangle$, 其中$S\subset G$为有限子集. 
        由讲义\textbf{例子 2.10}的3), $\langle S \rangle$具有如下描述:
        $$
        \langle S \rangle =\left\{s_1^{n_1}s_2^{n_2}\cdots s_k^{n_k} \mid k \in \N, s_i\in S, n_i\in \Z, s_i\neq s_{i+1} \right\}.
        $$
        由于下标$k$是可数的, $n_i$是可数的, 故$G$为可数集.
    \end{proof}

    \begin{problem}[C2)]
    \end{problem}
    \begin{proof}
        若$\Q=\langle q_1,q_2\cdots q_n \rangle$, 取$N$为各$q_i$分母的最大公倍数, 则所有生成元的分母整除$N$, 
        故生成群包含于$\frac{1}{N}\Z$. 但$\frac{1}{N+1}\notin \frac{1}{N}\Z$, 矛盾.
        因此$(\Q,+)$不是有限生成的.
    \end{proof}

    \begin{problem}[C3)]
    \end{problem}
    \begin{proof}
        设群$G$由有限子集$S=\{g_1,g_2,\ldots,g_n\}$生成, 即$G=\langle S \rangle$.
        由于$N\triangleleft G$为正规子群, 有自然的商映射
        $$
        \pi: G \to G/N,\ g \mapsto gN.
        $$
        由群同态的性质, $G/N=\langle \pi(S) \rangle$, 其中$\pi(S)=\{\pi(g_i)\mid g_i\in S\}$为有限子集.
        故$G/N$为有限生成的.
    \end{proof}

    \begin{problem}[C4)]
    \end{problem}
    \begin{proof}
        设$N$由有限子集$T=\{n_1,n_2,\ldots,n_m\}$生成, 即$N=\langle T \rangle$.
        设$G/N$由有限子集$S=\{g_1N,g_2N,\ldots,g_kN\}$生成, 即$G/N=\langle S \rangle$.
        取$S'=\{g_1,g_2,\ldots,g_k\}$, 则$S'$为$G$的有限子集. 
        
        下面证明$G=\langle S' \cup T \rangle$: 

        设$H=\langle S' \cup T \rangle$, 则$H\subset G$. 而对任意$g\in G$, 由$G/N=\langle S \rangle$, 有
        $$
        gN=(g_{i_1}N)^{n_1}\cdot(g_{i_2}N)^{n_2}\cdots(g_{i_k}N)^{n_k}=g_{i_1}^{n_1}g_{i_2}^{n_2}\cdots g_{i_k}^{n_k}N,
        \ \text{其中 } g_{i_j}\in S'\subset H,k, n_j\in \Z.
        $$
        因此, $g_{i_k}^{-n_k}\cdots g_{i_2}^{-n_2}g_{i_1}^{-n_1}g\in N=\langle T \rangle \subset H$, 故$g\in H \Rightarrow G\subset H$. 综上, $G=H=\langle S' \cup T \rangle$.
        
        因此, $G$为有限生成的.
    \end{proof}

    \begin{problem}[C5)]
    \end{problem}
    \begin{proof}
        注意到$G$的两个生成元都是上三角矩阵, 且上三角矩阵对于加减、乘法、取逆封闭, 故$G$中的元素均为上三角矩阵.

        先证明$H$是$G$的子群:
        \begin{itemize}
            \item[$-$] $H\neq \emptyset$, 因为单位元$I_2 \in H$.
            \item[$-$] $H$对乘法封闭:
                $\forall A_1,A_2\in H$, 有$A_1A_2\in G$, 且显然$A_1A_2$的对角线元素全为$1$, 故$A_1A_2\in H$.
            \item[$-$] $H$对取逆封闭:
                $\forall A\in H$, 有$A^{-1}\in G$, 且显然$A^{-1}$的对角线元素全为$1$, 故$A^{-1}\in H$.
        \end{itemize}
        综上, $H$为$G$的子群.   

        再证明$H$不是有限生成的:

        首先需要刻画$H$, 令$A=\begin{pmatrix} 2&0 \\ 0&1 \end{pmatrix}, B=\begin{pmatrix} 1&1 \\ 0&1 \end{pmatrix}$, 则$G=\langle A,B \rangle$.
        计算可知:
        $$
        A^n=\begin{pmatrix} 2^n&0 \\ 0&1 \end{pmatrix},\quad 
        A^{-n}=\begin{pmatrix} 2^{-n}&0 \\ 0&1 \end{pmatrix},\quad 
        A^{n}BA^{-n}=\begin{pmatrix} 1&2^n \\ 0&1 \end{pmatrix}, \ 
        \text{其中 } n\in \Z.
        $$
        故$\begin{pmatrix} 1&2^n \\ 0&1 \end{pmatrix}\in H$, 其中$n\in \Z$.

        反证法, 若$H$为有限生成的, 则存在$S=\left\{\begin{pmatrix} 1&n_1 \\ 0&1 \end{pmatrix},\begin{pmatrix} 1&n_2 \\ 0&1 \end{pmatrix},\cdots,\begin{pmatrix} 1&n_k \\ 0&1 \end{pmatrix}\right\}$, 其中$n_i\in \Z$, 使得$H=\langle S \rangle$. 
        而$\langle S \rangle\simeq \langle n_1, n_2, \cdots n_k\rangle_{(\Q,+)}$, 
        因为$$\begin{pmatrix} 1&a \\ 0&1 \end{pmatrix}\begin{pmatrix} 1&b \\ 0&1 \end{pmatrix}=\begin{pmatrix} 1&a+b \\ 0&1 \end{pmatrix},\quad
        \begin{pmatrix} 1&a \\ 0&1 \end{pmatrix}^{-1}=\begin{pmatrix} 1&-a \\ 0&1 \end{pmatrix},\quad a,b\in \Q.$$

        设$N$为所有$n_i$的分母的最大公倍数, 则对$H$中的任意元素形如$\begin{pmatrix} 1&2^{-n} \\ 0&1 \end{pmatrix}, n>N$, 
        因为$2^{-n}$的分母不整除$N$, 故$2^{-n}\notin \langle n_1, n_2, \cdots n_k\rangle_{(\Q,+)}$, 故$\begin{pmatrix} 1&2^{-n} \\ 0&1 \end{pmatrix}\notin \langle S \rangle$. 
        矛盾. 因此, $H$不是有限生成的.

        综上, 有限生成的群的子群不一定是有限生成的.
    \end{proof}

    \begin{problem}[C6)]
    \end{problem}
    \begin{proof}
        \begin{itemize}
            \item[$\bullet$] 
                首先$Hg_i^{-1}$显然是$H$的一个右陪集. 
                而$Hg_i^{-1}=Hg_j^{-1}\Leftrightarrow g_ig_j^{-1}\in H$, 证明与$g_iH=g_jH\Leftrightarrow g_ig_j^{-1}$类似, 这里不再赘述. 由$\left\{g_iH\mid i\in I, g_i\in G\right\}$是$H$在$G$中所有左陪集的集合知, $\left\{Hg_i^{-1}\mid i\in I, g_i\in G\right\}$中的元素两两不交.
                
                下面证明$\left\{Hg_i^{-1}\mid i\in I. g_i\in G\right\}$包含所有右陪集. $\forall g\in G, \exists i\in I, \st g^{-1}\in g_iH$, 即$\exists h\in H,\st g^{-1}=g_ih$. 
                从而$g=h^{-1}g_i^{-1}\in Hg_i^{-1}$. 故$\left\{Hg_i^{-1}\mid i\in I, g_i\in G\right\}$包含所有右陪集.

                综上, $\left\{Hg_i^{-1}\mid i\in I, g_i\in G\right\}$是所有右陪集的集合.
            \item[$\bullet$] 
                假设有限子集$S$生成$G$, $\left\{g_iH\mid i\in I, g_i\in G\right\}$是$H$的所有左陪集, 由$[G:H]<\infty$, 知$I$为有限指标集. 由上一小问知$\left\{Hg_i^{-1}\mid i\in I, g_i\in G\right\}$是所有右陪集. 并且取左陪集代表元集合$\{g_i\mid i\in I\}$, 不妨取$g_1=1_{_G}$.

                考虑:
                $$
                T=\left\{xsy\mid x,y\in \left\{g_i,g_i^{-1}\mid i\in I\right\},s\in S\right\}\cap H
                $$
                由于$I,S$都是有限集, 于是$T$也是有限集.
                $\forall h\in H$, 有$h=s_1s_2\cdots s_k$, 其中$s_i\in S\cup S^{-1}$
                对于$s_1, \exists i_1\in I,h_1\in H,\st$
                $$
                s_1=hg_{i_1}^{-1} \Rightarrow s_1g_{i_1}=h_1\in H,
                $$ 
                而对于$s_j(2\leq j\leq k), \exists i_j\in I,h_j\in H,\st$
                $$
                g_{i_{j-1}}^{_1}s_j=hg_{i_j}^{-1} \Rightarrow g_{i_{j-1}}^{-1}s_jg_{i_j}=h_j\in H.$$
                从而有:
                \begin{align*}
                    h&=(s_1g_{i_1})(g_{i_1}^{-1}s_2g_{i_2})\cdots(g_{i_{j-1}}^{-1}s_jg_{i_j})\cdots(g_{i_{k-1}}^{-1}s_kg_{i_k})g_{i_k}^{-1} \\
                    &=(g_1^{-1}s_1g_{i_1})(g_{i_1}^{-1}s_2g_{i_2})\cdots(g_{i_{j-1}}^{-1}s_jg_{i_j})\cdots(g_{i_{k-1}}^{-1}s_kg_{i_k})g_{i_k}^{-1},
                \end{align*}
                而$h$与右边前$k$个元素都在$H$中,所以$g_{i_k}^{-1}\in H \Rightarrow g_{i_k}=1_{_G}$.
                
                所以$H=\langle T \rangle$, $H$是有限生成的.
        \end{itemize}
    \end{proof}

\newpage

\section{D. 线性群中元素的阶的几个命题}

    \begin{problem}[D1)]
    \end{problem}
    
    \begin{proof}
        \begin{itemize}
            \item[$\bullet$]
                $\forall A \in \textbf{GL}(n,\Z)$, 由于$A\in \textbf{M}_n(\Z)$, $\det(A)\in \Z$,
                而由于,又$\det(A)\det(A^{-1})=1$, 故而$\det(A)=\pm1$.
            \item[$\bullet$]
                设$A$的两个特征值为$\lambda, \mu$, 
            \item[$\bullet$] 
        \end{itemize} 
    \end{proof}

\end{document}
