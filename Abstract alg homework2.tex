\documentclass[a4paper, 12pt, UTF8, heading=true, scheme=chinese]{ctexart}

\usepackage[a4paper, left=2cm, right=2cm, top=2cm, bottom=2cm]{geometry}
\usepackage{amsmath,amssymb,bm,graphicx,xcolor,tikz,array,booktabs,multicol,multirow,titlesec,hyperref,biblatex,algorithm2e,listings}

\titleformat{\section}
  {\normalfont\Large\bfseries\raggedright}
  {}
  {0pt}
  {}

\newcommand{\R}{\mathbb{R}}
\newcommand{\Z}{\mathbb{Z}}
\newcommand{\N}{\mathbb{N}}
\newcommand{\Q}{\mathbb{Q}}
\newcommand{\C}{\mathbb{C}}
\newcommand{\st}{\text{s.t.}}

\newenvironment{solution}{\par\noindent\textbf{解:}}{\hfill$\square$\par}
\newenvironment{proof}{\par\noindent\textbf{证明:}}{\hfill$\square$\par}
\newenvironment{problem}[1][]{\par\noindent\textmd{#1}\par}{}
\newenvironment{note}{\par\noindent\textbf{注:}}{\par}
\newenvironment{tip}{\par\noindent\textbf{提示:}}{\par}

\linespread{1.5}

\title{\textbf{群 与 Galois 理论 \\ 作业2}}
\author{陈宏泰 \\ 清华大学数学科学系 \\ \texttt{cht24@mails.tsinghua.edu.cn}}
\date{\today} 

\begin{document}

\maketitle

\tableofcontents
\newpage

\section{A. 对称群\texorpdfstring{$\mathfrak{S}_n$}{S\_n} 中的计算}

    \begin{problem}[A1)]   
    \end{problem}

    \begin{solution}
        由上课命题知, $\mathfrak{S}_n$可以由$\{(1,k)\mid k=2,\cdots,n\}$或者$\{(k,k+1)\mid k=1\cdots,n-1\}$生成.
        于是有$|S|_\text{min}\leq n-1$.

        下面说明$|S|$不能小于等于$n-2$: 假设$|S|=n-2$, 那么可以定义集合$N=\{1,2,\cdots,n\}$中的等价关系“$\sim$”:
        $$
        i\sim j \Leftrightarrow \exists\ \sigma\in \langle S\rangle,\  \st\  \sigma(i)=j.
        $$

        可知$\mathfrak{S}_n$能够被$S$生成$\Rightarrow$集合$N$仅有一个等价类. 于是考察$N$中的等价类.
        
        任取$i_1\in N$, 如果$\exists\ (i_1,i_2) \in S, i_1\neq i_2\in N$, 那么知$i_1\sim i_2$.
        如果不存在, 那么$i_1$无法与其他元素置换, 那么知$N$有超过一个等价类, 矛盾.
        如果是前一种情况, 则继续. 如果$\exists\ (i_k,i_3)\in S, k=1,2, i_k\neq i_3\in S$, 那么$i_k\sim i_3$.
        如果不存在, 那么$i_1,i_2$无法与其他元素置换, 那么知$N$有超过一个等价类, 矛盾.
        以此类推, 如果出现不存在的情况, 结论成立. 如果都存在, 那么$i_1$的等价类中至多有$(n-2)+1=n-1$个元素,
        因为$S$中每个对换至多向等价类中增加一个元素. 那么$N$中仍然会剩余一个元素不在此等价类中, 即$N$中有超过一个等价类, 矛盾.
        
        如果$|S|\leq n-2$以上过程都能导出矛盾. 综上, $|S|$的最小值是$n-1$.
    \end{solution}

    \begin{note}
        用图论的观点来看, $n-2$条线无法连接$n$个点并使之连通.
    \end{note}

    \begin{problem}[A2)]
    \end{problem}

    \begin{proof}
        假设$|i_0-j_0|$与$n$互素, 不妨设$i_0<j_0$, 设$m=j_0-i_0$, 则$m$与$n$互素. 对换$(i_0,j_0)$可以写作$(i_0,i_0+m)$. 
        
        通过以下变换:
        $$
        (1,2,\cdots,n)^{-k}(i_0,i_0+m)(1,2,\cdots,n)^{k},
        $$
        可以得到所有形如$(i_0+k,i_0+k+m),k\in \Z$的对换, 其中可以利用$i_0+k\equiv j \pmod n, j\in S$, 将$i_0+k$与$j$等同起来.
        由于$m$与$n$互素, 存在整数$a$与$b$, $\st\ am+bn=1$. 于是又可以通过
        \begin{align*}
            &(i_0+(a-1)m,i_0+am)(i_0+(a-2)m,i_0+(a-1)m)\cdots(i_0,i_0+m)\\
            &(i_0+m,i_0+2m)\cdots(i_0+(a-1)m,i_0+am),
        \end{align*}
        得到$(i_0,i_0+am)=(i_0,i_0+1)$. 又可以通过
        $$
        (1,2,\cdots,n)^{1-i_0}(i_0,i_0+1)(1,2,\cdots,n)^{i_0-1},
        $$
        得到$(1,2)$, 由上课命题知$\{(1,2),(1,2,\cdots,n)\}$生成$\mathfrak{S}_n$, 
        于是$S_4=\{(i_0,j_0),(1,2,\cdots,n)\}$生成$\mathfrak{S}_n$.
    \end{proof}

    \begin{problem}[A3)]
    \end{problem}

    \begin{proof}
        假设$\mathfrak{A}_n,n\geq 5$有两个不同的三循环$\alpha=(i_1,i_2,i_3),\beta=(j_1,j_2,j_3)$.
        
        如果$\{i_1,i_2,i_3\}\cap\{j_1,j_2,j_3\}=\emptyset$, 那么由
        \begin{align*}
            \beta=&(i_1,j_1,j_2)(i_2,j_1,j_3)(i_3,j_1,j_3)\alpha(i_3,j_3,j_1)(i_2,j_3,j_1)(i_1,j_2,j_1)\\
            =&((i_1,j_1,j_2)(i_2,j_1,j_3)(i_3,j_1,j_3))\alpha((i_1,j_1,j_2)(i_2,j_1,j_3)(i_3,j_1,j_3))^{-1},
        \end{align*}
        知$\alpha$与$\beta$共轭.
        
        如果$\{i_1,i_2,i_3\}\cap\{j_1,j_2,j_3\}\neq\emptyset$, 且如果有一个相同的元素, 不妨设$i_1=j_1$, 那么由
        \begin{align*}
        \beta&=(i_3,j_3)(i_2,j_2)\alpha(i_2,j_2)(i_3,j_3)\\
        &=((i_2,j_2)(i_3,j_3))\alpha((i_2,j_2)(i_3,j_3))^{-1},
        \end{align*}
        知$\alpha$与$\beta$共轭.

        如果有两个相同的元素, 不妨设$i_1=j_1,i_2=j_2$或者$i_1=j_1,i_2=j_3$. 
        当$i_1=j_1,i_2=j_3$时, 有
        \begin{align*}
        \beta&=(i_1,i_2)(i_3,j_2)\alpha(i_3,j_2)(i_1,i_2)\\
        &=((i_1,i_2)(i_3,j_2))\alpha((i_1,i_2)(i_3,j_2))^{-1},
        \end{align*}
        知$\alpha$与$\beta$共轭. 当$i_1=j_1,i_2=j_2$时, 由$n\geq 5$知, 存在$k\in \{1,2,\cdots,n\}\backslash\{i_1,i_2,i_3,j_3\}$,
        那么有
        \begin{align*}
        \beta&=(i_3,j_3,k)\alpha(i_3,k,j_3)\\
        &=(i_3,j_3,k)\alpha(i_3,j_3,k)^{-1},
        \end{align*}
        知$\alpha$与$\beta$共轭.

        综上, $\mathfrak{A}_n,n\geq 5$中的任意两个三循环共轭.
    \end{proof}

    \begin{note}
        可以直接通过3-循环在$\mathfrak{S}_n$共轭性质进行证明, 但需要注意共轭元是否在$\mathfrak{A}_n$中. 
        如果不在其中, 则需要引入一个与$\beta$不交的对换来保证, 而由于$n\geqslant 5$, 一定能取到这样的对换.
    \end{note}

    \begin{problem}[A4)]
    \end{problem}

    \begin{proof}
        由课上命题已知, $\mathfrak{A}_5$可以由3-循环子集生成. 下面说明任意一个3-循环可以由双对换生成:
        设$\alpha=(i,j,k)$, 则$\exists\ l\neq m\in \{1,2,3,4,5\}\backslash\{i,j,k\}$, 那么有
        $$
        \alpha=(i,j)(j,k)=(i,j)(l,m)(l,m)(j,k)=((i,j)(l,m))((l,m)(j,k)),
        $$
        于是$\mathfrak{A}_5$可以由双对换生成.
    \end{proof}

    \begin{problem}[A5)]
    \end{problem}

    \begin{proof}
        a)
        如果$y\in G$是一个3阶元素, 则$y$是一个3-循环.
        充分性: 不妨设$y=(i,j,5)$, 其中$i\in\{1,2\},j\in\{3,4\}$.($j\in\{1,2\},i\in\{3,4\}$时同理) 则
        $$
        xy=(12)(34)(i,j,5)=(12)(k,j,5,i)=(l,i,k,j,5),
        $$
        其中$k\in\{1,2\}\backslash\{i\},l\in\{3,4\}\backslash\{j\}$.从而$xy$是一个5-循环, $\text{ord}(xy)=5$.

        必要性: 如果$y$的两个不动点不是一个在$\{1,2\}$中, 另一个在$\{3,4\}$中, 
        那么$y$的两个不动点要么都在$\{1,2\}$中, 要么都在$\{3,4\}$中. 
        不妨设$y$的两个不动点都在$\{1,2\}$中, 于是$y$与$(12)$交换,
        那么由$xy=(12)(34)y$, 知$(xy)^5=(12)^5((34)y)^5$. 显然$(12)^5=(12)$, $((34)y)^5\neq(12)$.
        于是$(xy)^{-5}\neq 1$, $\text{ord}(xy)\neq 5$, 矛盾.

        b) 
        显然有
        $u$的阶为2$\ \Leftrightarrow u = (ij)(kl)$, 其中$i,j,k,l$互不相同.
        $v$的阶为3$\ \Leftrightarrow v$为3-循环.
        由a)知
        $$
        \text{ord}(uv)=5\ \Leftrightarrow \text{$v$的两个不动点一个在$\{i,j\}$中, 另一个在$\{k,l\}$中.}
        $$
        对$X$中元素计数, $u$有$\frac{C_5^2C_3^2}{2}=15$种取法,
        而对于每个$u$, $v$有$2\times 2\times 2=8$种取法, 故$X$中共有$15\times 8=120$个元素, $|X|=120$.

        c) 
        定义$\textbf{Aut}(G)$在$X$上的作用为:
        \begin{align*}
        \textbf{Aut}(G)\times X &\to X\\
        (\varphi,(u,v))&\mapsto\varphi\cdot (u,v)=(\varphi(u),\varphi(v)),
        \end{align*}
        由于$\varphi\in \textbf{Aut}(G)$是一个同构, 
        故$\text{ord}(\varphi(u))=\text{ord}(u)$,$\text{ord}(\varphi(v))=\text{ord}(v)$,$\text{ord}(\varphi(u)\varphi(v))=\text{ord}(\varphi(uv))=\text{ord}(uv)$.
        于是$\varphi\cdot (u,v)\in X$. 
        $\forall\ \varphi,\psi\in \textbf{Aut}(G)$, 有
        $$
        (\varphi\cdot \psi)\cdot (u,v)=(\varphi\psi(u),\varphi\psi(v))=\varphi\cdot (\psi\cdot (u,v)),
        $$
        又$\textbf{Aut}(G)$中单位元1,满足$1(u,v)=(u,v)$,于是$\textbf{Aut}(G)$在$X$上的作用是群作用.

        $\forall\ (u,v)\in X$, 若$\varphi\cdot (u,v)=(u,v)$, 则$\varphi(u)=u,\varphi(v)=v$.
        考虑$S=\langle u,v\rangle$, 由于$u,v,uv\in S$的阶分别为2,3,5,
        故$\text{lcm}(2,3,5)=30\big| |S|$. 又$|S|\big| |G|$, 而$|G|=60$,
        故$|S|=30$或者$60$. 知$S$是$G$的一个正规子群, 所以$v$的共轭类也在$S$中.
        由A3)知, $\mathfrak{A}_5$中所有3-循环构成一个共轭类, 故$S$包含所有3-循环.
        由于3-循环生成$\mathfrak{A}_5=G$, 故$S=G$. 
        于是$\varphi$在作用在$S=G$上是恒等映射,从而$\textbf{Aut}(G)$在$X$上的作用是自由的.

        据此, 有$|\textbf{Aut}(G)|\leq|X|=120$. 由于$\textbf{Aut}(G)$包含$\mathfrak{S}_5$, 从而$\textbf{Aut}(G)=\mathfrak{S}_5$.
    \end{proof}

    \begin{note}
        如果知道$B$题的结论, 则可以直接利用$\mathfrak{A}_5$是单群与Lagrange定理来证明A5)c).
    \end{note}

    \begin{problem}[A6)]
    \end{problem}

    \begin{proof}
        令$G=\mathfrak{A}_4$.

        a)
        假设$x=(12)(34)\in G$, $y\in G$是一个3-循环. 则$xy$是一个3-循环.
        不妨设$y=(i,j,k)$, 其中$i\neq j\in\{1,2\},k\in\{3,4\}$.($i\neq j\in\{3,4\},k\in\{1,2\}$时同理) 则
        $$
        xy=(12)(34)(i,j,k)=(34)(12)(i,j)(j,k)=(34)(j,k)=(l,k,j),
        $$
        其中$l\in\{3,4\}\backslash\{k\}$.从而$xy$是一个3-循环, $\text{ord}(xy)=3$.

        b) 
        令$X=\{(u,v)\in G\times G\mid u,v,uv\text{的阶分别为2,3,3}\}$.

        显然有
        $u$的阶为2$\ \Leftrightarrow u = (ij)(kl)$, 其中$i,j,k,l$互不相同.
        $v$的阶为3$\ \Leftrightarrow v$为3-循环.
        由a)知$\text{ord}(uv)=3$,
        对$X$中元素计数, $u$有$C_4^2/2=3$种取法,
        而对于每个$u$, $v$有$2\times C_4^3=8$种取法, 故$X$中共有$3\times 8=24$个元素, $|X|=24$.
        
        c) 
        定义$\textbf{Aut}(G)$在$X$上的作用为:
        \begin{align*}
        \textbf{Aut}(G)\times X &\to X\\
        (\varphi,(u,v))&\mapsto\varphi\cdot (u,v)=(\varphi(u),\varphi(v)),
        \end{align*}
        由于$\varphi\in \textbf{Aut}(G)$是一个同构, 
        故$\text{ord}(\varphi(u))=\text{ord}(u)$,$\text{ord}(\varphi(v))=\text{ord}(v)$,$\text{ord}(\varphi(u)\varphi(v))=\text{ord}(\varphi(uv))=\text{ord}(uv)$.
        于是$\varphi\cdot (u,v)\in X$. 
        $\forall\ \varphi,\psi\in \textbf{Aut}(G)$, 有
        $$
        (\varphi\cdot \psi)\cdot (u,v)=(\varphi\psi(u),\varphi\psi(v))=\varphi\cdot (\psi\cdot (u,v)),
        $$
        又$\textbf{Aut}(G)$中单位元1,满足$1(u,v)=(u,v)$,于是$\textbf{Aut}(G)$在$X$上的作用是群作用.

        $\forall\ (u,v)\in X$, 若$\varphi\cdot (u,v)=(u,v)$, 则$\varphi(u)=u,\varphi(v)=v$,
        考虑$S=\langle u,v\rangle$, 由于$u,v,uv\in S$的阶分别为2,3,3,
        故$\text{lcm}(2,3)=6\big| |S|$. 又$|S|\big| |G|$, 而$|G|=12$,
        故$|S|=6$或者$12$. 首先注意到$1,u,v,v^2,uv,(uv)^2$互不相同, 而以a)中的$u,v$为例,有
        $$
        vu=(i,j,k)(12)(34)=(i,j)(j,k)(12)(34)=(i,j)(j,k,l)(12)=(i,j)(k,l,j,i)=(k,l,i),
        $$
        其中$l\in\{3,4\}\backslash\{k\}$. 显然$vu\neq uv$且$vu\neq (uv)^2$. 于是$|S|>6$.
        从而$|S|=12$, 则$S=G$. 于是$\varphi$在作用在$S=G=\mathfrak{A}_4$上是恒等映射, 
        从而$\textbf{Aut}(G)$在$X$上的作用是自由的.

        据此, 有$|\textbf{Aut}(G)|\leq|X|=24$. 由于$\textbf{Aut}(G)$包含$\mathfrak{S}_4$, 从而$\textbf{Aut}(\mathfrak{A}_4)=\mathfrak{S}_4$.
    \end{proof}

    \begin{problem}[A7)]
    \end{problem}

    \begin{proof}
        令$G=\mathfrak{S}_4$.

        a)
        假设$x=(12)\in G$, $y\in G$是一个3阶元素, 则$y$是一个3-循环. 
        下面证明: $xy$的阶为4,当且仅当$y$的不动点在$\{1,2\}$之中.
        充分性: 不妨设$y=(i,3,4)$, 其中$i\in\{1,2\}$.($y=(i,4,3)$时同理) 则
        $$
        xy=(12)(i,3,4)=(j,i,3,4),
        $$
        其中$j\in\{1,2\}\backslash\{i\}$.从而$xy$是一个4-循环, $\text{ord}(xy)=4$.

        必要性: 如果$y$的不动点在$\{1,2\}$中, 那么$y$的不动点在$\{3,4\}$中. 
        不妨设$y=(1,2,i)$, 其中$i\in\{3,4\}$,($y=(2,1,i)$时同理) 则
        $$
        xy=(12)(1,2,i)=(12)(12)(2,i)=(2,i),
        $$
        那么由$\text{ord}(xy)=2$, 矛盾.

        b) 
        令$X=\{(u,v)\in G\times G\mid u,v,uv\text{的阶分别为2,3,4}\}$.

        显然有
        $u$的阶为2$\ \Leftrightarrow u = (ij)$, 其中$i,j$互不相同.
        $v$的阶为3$\ \Leftrightarrow v$为3-循环.
        由a)知
        $$
        \text{ord}(uv)=4\ \Leftrightarrow \text{$v$的不动点在$\{i,j\}$中.}
        $$
        对$X$中元素计数, $u$有$C_4^2=6$种取法,
        而对于每个$u$, $v$有$2\times 2=4$种取法, 故$X$中共有$6\times 4=24$个元素, $|X|=24$.

        c) 
        定义$\textbf{Aut}(G)$在$X$上的作用为:
        \begin{align*}
        \textbf{Aut}(G)\times X &\to X\\
        (\varphi,(u,v))&\mapsto\varphi\cdot (u,v)=(\varphi(u),\varphi(v)),
        \end{align*}
        由于$\varphi\in \textbf{Aut}(G)$是一个同构, 
        故$\text{ord}(\varphi(u))=\text{ord}(u)$,$\text{ord}(\varphi(v))=\text{ord}(v)$,$\text{ord}(\varphi(u)\varphi(v))=\text{ord}(\varphi(uv))=\text{ord}(uv)$.
        于是$\varphi\cdot (u,v)\in X$. 
        $\forall\ \varphi,\psi\in \textbf{Aut}(G)$, 有
        $$
        (\varphi\cdot \psi)\cdot (u,v)=(\varphi\psi(u),\varphi\psi(v))=\varphi\cdot (\psi\cdot (u,v)),
        $$
        又$\textbf{Aut}(G)$中单位元1,满足$1(u,v)=(u,v)$,于是$\textbf{Aut}(G)$在$X$上的作用是群作用.

        $\forall\ (u,v)\in X$, 若$\varphi\cdot (u,v)=(u,v)$, 则$\varphi(u)=u,\varphi(v)=v$.
        考虑$S=\langle u,v\rangle$, 由于$u,v,uv\in S$的阶分别为2,3,4,
        故$\text{lcm}(2,3,4)=12\big| |S|$. 又$|S|\big| |G|$, 而$|G|=24$,
        故$|S|=12$或者$24$. 知$S$是$G$的一个正规子群, 所以$v$的共轭类也在$S$中.
        而$\mathfrak{S}_4$中所有3-循环构成一个共轭类, 故$S$包含所有3-循环.
        又$1,u,uv,(uv)^2,(uv)^3\in S$, 互不相同, 于是$|S|>12$.
        故$|S|=24,S=G$. 
        于是$\varphi$在作用在$S=G=\mathfrak{S}_4$上是恒等映射,
        从而$\textbf{Aut}(G)$在$X$上的作用是自由的.

        据此, 有$|\textbf{Aut}(G)|\leq|X|=24$. 由于$\textbf{Aut}(G)$包含$\mathfrak{S}_4$, 从而$\textbf{Aut}(\mathfrak{S}_4)=\mathfrak{S}_5$.
    \end{proof}

    \begin{problem}[A8)]
    \end{problem}

    \begin{proof}
        一方面, $\textbf{Aut}(\mathfrak{S}_4)$包含$\mathfrak{S}_4$的内自同构群$\textbf{Int}(\mathfrak{S}_4)$, 而$\mathfrak{S}_4$的中心是一个平凡群, 
        故$\mathfrak{S}_4\cong \textbf{Int}(\mathfrak{S}_4)/Z(\mathfrak{S}_4)=\textbf{Int}(\mathfrak{S}_4)\triangleleft \textbf{Aut}(\mathfrak{S}_4)$.

        另一方面, $\forall\ \varphi\in\textbf{Aut}(\mathfrak{S}_4)$, 下面说明其将对换映射到对换:
        设$\sigma=(i,j)$是一个对换, 由$\sigma^2=1$知, $\varphi(\sigma)^2=1$, 故$\varphi(\sigma)$的可能类型只有三种: 单位元, 对换, 双对换.
        如果$\varphi(\sigma)$是单位元, 则$\varphi$不是单射, 矛盾.
        如果$\varphi(\sigma)$是双对换, 不妨设$\varphi(\sigma)=(k,l)(m,n)$, 那么取
        $$\rho=
        \begin{pmatrix} 
        k & l & m & n\\
        i & j & p & q
        \end{pmatrix},
        $$
        其中$p,q\in \{1,2,3,4\}\backslash\{i,j\}$, 则有
        $$
        \varphi(\rho\sigma\rho^{-1})=\varphi(\rho)\varphi(\sigma)\varphi(\rho)^{-1}=(i,j)(p,q),
        $$
        但$\rho\sigma\rho^{-1}=(k,l)$是一个对换, 故$\sigma$将所有对换映射到双对换. 但两者的数量不同, 与$\varphi$是自同构矛盾.
        于是$\varphi(\sigma)$是一个对换, 即得$\varphi$将对换映射到对换.
        又由于对换生成$\mathfrak{S}_4$, 故$\varphi$也将偶置换映射到偶置换, 所以$\varphi\in \textbf{Aut}(\mathfrak{A}_4)$.
        于是$\textbf{Aut}(\mathfrak{S}_4)\subset \textbf{Aut}(\mathfrak{A}_4)=\mathfrak{S}_4$.

        综上, 有$\textbf{Aut}(\mathfrak{S}_4)=\mathfrak{S}_4$.
    \end{proof}

    \begin{note}
        对于所有$n\geqslant 3,n\neq 6$的情况, 都可以使用以上的方法证明$\textbf{Aut}(\mathfrak{S}_4)=\textbf{Aut}(\mathfrak{A}_4)=\mathfrak{S}_4$.
    \end{note}

\newpage

\section{B. 交替群\texorpdfstring{$\mathfrak{A}_n\ (n\geqslant 5)$}{A\_n} 是单群}

    \begin{problem}[B0)]   
    \end{problem}

    \begin{solution}
        $\mathfrak{A}_3$的正规子群有$\{e\},\mathfrak{A}_3$.

        \ $\mathfrak{A}_4$的正规子群有$\{e\},\{e,(12)(34),(13)(24),(14)(23)\},\mathfrak{A}_4$.
    \end{solution}

    \begin{problem}[B1)]
    \end{problem}

    \begin{proof}
        B1-1)
        注意到
        $$
        \tau=(34)(25)\sigma(25)(34)=((34)(25))\sigma((34)(25))^{-1},
        $$
        于是$\sigma$与$\tau$在$\mathfrak{A}_5$中共轭.

        特别地, $\sigma\tau=(152)$是一个3-循环.

        B1-2)
        注意到
        $$
        \tau=(123)\sigma(123)=(123)\sigma(123)^{-1},
        $$
        于是$\sigma$与$\tau$在$\mathfrak{A}_5$中共轭.

        特别地, $\tau\sigma^2=(253)$是一个3-循环.

        B1-3)
        只需要证明$N$中有一个3-循环即可, 由$\mathfrak{A}_5$中所有3-循环共轭可知$N$包含所有3-循环.
        而$\mathfrak{A}_5$中的元素只有单位元, 3-循环, 双对换, 5-循环四种可能的类型. 由$N$非平凡可知,
        如果$N$中有一个3-循环, 结论成立. 
        如果$N$中有一个双对换, 由B1-1)知$N$中有一个3-循环, 结论成立.
        如果$N$中有一个5-循环, 由B1-2)知$N$中有一个3-循环, 结论成立.
        综上, $N$包含所有的3-循环, 从而, $\mathfrak{A}_5$是单群。
    \end{proof}

    \begin{problem}[B2)]
    \end{problem}

    \begin{proof}
        B2-1)
        对于$\mathfrak{A}_5$中单位元1, 共轭类元素个数为1. 对于3-循环, 由A3)知共轭类元素个数为$C_5^3\times 2=20$.
        
        对于双对换, 任取两个不同的双对换$\sigma=(i_1,i_2)(i_3,i_4),\tau=(j_1,j_2)(j_3,j_4)$, 取
        $$
        \rho=
        \begin{pmatrix}
        i_1 & i_2 & i_3 & i_4 & i_5\\
        j_1 & j_2 & j_3 & j_4 & j_5
        \end{pmatrix}
        $$
        其中$i_5,j_5$分别为$\{1,2,3,4,5\}\backslash\{i_1,i_2,i_3,i_4\},\{1,2,3,4,5\}\backslash\{j_1,j_2,j_3,j_4\}$,
        则$\tau=\rho\sigma\rho^{-1}$. 
        如果$\rho$是奇置换, 则取$\rho'=(i_1,i_2)\rho$, 则$\rho'$是偶置换, 且$\tau=\rho'\sigma(\rho')^{-1}$;
        如果$\rho$是偶置换, 则直接取$\rho'=\rho$. 于是$\sigma$与$\tau$在$\mathfrak{A}_5$中共轭.
        于是双对换构成一个共轭类. 对于双对换, 共轭类有$\frac{C_5^2C_3^2}{2}=15$个元素.
 
        对于5-循环, 共轭类元素个数为$(5-1)! = 24$. 设$\tau=(12345),\sigma$是两个不同的5-循环,
        则存在$\rho\in \mathfrak{S}_5$, 使得$\sigma=\rho\tau\rho^{-1}=(\rho(1),\rho(2),\rho(3),\rho(4),\rho(5))$.
        可知$\sigma$与$\tau$在$\mathfrak{A}_5$中共轭的充分必要条件是$\rho$的逆序数为偶数, 同时, 这样的$\rho$有60个, 生成的$\sigma$有$60/5=12$个.
        同理可知对应$\rho$的逆序数为奇数的5-循环$\varsigma$也为一个共轭类. 这两个共轭类的元素数量都为12.
        
        综上, $\mathfrak{A}_5$的共轭类有5个, 并且每个共轭类中的元素个数分别为1,12,12,15和20.

        B2-2)
        如果子群$N\subset \mathfrak{A}_5$在$\mathfrak{A}_5$的共轭下不变, 那么共轭类要么全部包含在$N$中, 要么全部不包含在$N$中.
        由B2-1)知, $\mathfrak{A}_5$的共轭类大小分别为1,12,12,15,20. 进行组合, 有$|N|$可能的取值为
        1,13,16,21,25,28,33,36,40,45,48和60

        B2-3)
        假设$N$是$\mathfrak{A}_5$的一个非平凡正规子群.
        
        由B2-2)知$|N|$的可能取值为13,16,21,25,28,33,36,40,45和48.
        由于$N$是$\mathfrak{A}_5$的子群, 故$|N|\big| 60$. 于是$|N|$的可能取值为2,3,4,5,6,10,15,20和30.
        综上, $N$的阶数没有可能的取值, 故$\mathfrak{A}_5$没有非平凡子群, $\mathfrak{A}_5$是单群.
        \end{proof}

    \begin{problem}[B3)]
    \end{problem}

    \begin{proof}
        B3-1)

        如果存在$\sigma\in N-\{1\}$, 使得$\sigma(n)=n$, 可以将$\sigma$看作$\mathfrak{A}_{n-1}$中的一个元素.
        $\forall\ \rho\in \mathfrak{A}_{n-1}$, 由$\mathfrak{A}_{n-1}\subset \mathfrak{A}_n$知,
        可以将$\rho$看作$\mathfrak{A}_n$中的一个元素, 其中$\rho(n)=n$.

        考虑集合$S_n=\{\rho\sigma^k\rho^{-1}\mid\sigma\neq1, k\in\Z, \rho\in\mathfrak{A}_{n-1}\}\subset\mathfrak{A}_{n-1}$.
        容易验证, $S_n$是$\mathfrak{A}_{n-1}$的一个正规子群. 由$\mathfrak{A}_{n-1}$的单群性质, 知$S_n=\mathfrak{A}_{n-1}$.
        同时, 由于$N$是$\mathfrak{A}_n$的正规子群, $\rho\sigma\rho^{-1}\in N$, 从而$S_n\subset N$. 于是$S_n=\mathfrak{A}_{n-1}$是$N$的子群.
        
        而$\forall\ k\in\{1,2,\cdots,n-1\}, \exists\ \varsigma \in \mathfrak{A}_{n-1}, \st\ \varsigma(k)=k$, 
        因为所有3-循环在$\mathfrak{A}_{n-1}$中.
        用完全相同的方法可以证明, $S_k=\{\varsigma\mid \varsigma(k)=k\}\simeq \mathfrak{A}_{n-1}\subset N$是$N$的子群.
        于是知道$\mathfrak{A}_n$中的每个3-循环都在$N$中, 从而$N=\mathfrak{A}_n$.
        
        \begin{tip}
            $n$只是一个symbol.
        \end{tip}

        B3-2)

        注意到$\tau\sigma\tau^{-1}\sigma\in N$
        $$
        \tau\sigma\tau^{-1}\sigma(n)=\tau\sigma(n)=n,
        $$
        而由B3-1)知, 不存在$\rho\in N-\{1\}$, 使得$\rho(n)=n$. 所以$\tau\sigma\tau^{-1}\sigma=1$.

        B3-3)

        首先,由B3-1)知,不存在$k\in\{1,2,\cdots,n\}$,使得$\sigma(k)=k$.又由于B3-2)知$\tau\sigma\tau^{-1}\sigma=1$, 

        考虑$\sigma^2$, 如果$\exists\ m\notin\{i,j,n,\sigma(n)\},\st\ \sigma(m)\notin \{i,j,n,\sigma(n)\}$, 那么
        $$
        \tau\sigma\tau^{-1}\sigma(m)=\tau\sigma\tau^{-1}(\sigma(m))=\tau\sigma(\sigma(m))=\tau(\sigma^2(m))=m,
        $$
        那么, $\sigma^2(m)=\tau^{-1}(m)=m$, 得出$\sigma^2$有不动点, 即$\sigma^2=1$.

        如果$\forall\ m\notin\{i,j,n,\sigma(n)\},\sigma(m)\in \{i,j,n,\sigma(n)\}$, 并且容易观察到此时$n\leqslant 7$(并不需要用到).
        那么由于$n\geqslant 6$, 必定$\exists\ m\notin\{i,j,n,\sigma(n)\},\st\ \sigma(m)\neq n$. 
        那么取${i',j'},\st\ m,\sigma(m)\notin\{i',j'\}$. 那么归为上一种情况, 知$\sigma^2=1$.

        由此可知, $\sigma^{-1}\tau\sigma\tau^{-1}=1$, 于是有$\tau\sigma=\sigma\tau$.
        固定$i,j$后, 有
        $$
        \tau\sigma(i)=\sigma\tau(i)=\sigma(j).
        $$
        可知$\sigma(i),\sigma(j)\in\{i,j,n,\sigma(n)\}$. 显然有$\sigma(i)\neq i,\sigma(n)$, $\sigma(j)$同理.
        而如果$\sigma(i)=n$, 那么$\tau\sigma(i)=\tau(n)=\sigma(n)=\sigma(j)$, 矛盾.
        故$\sigma(i)=j$, 同理$\sigma(j)=i$, 即$\sigma:\{i,j\}\to\{i,j\}$.

        % 当$n=7$时, 此时一定有

        % $\forall\ l\in\{i,j,n,\sigma(n)\}$, 如果有$\sigma(l)\notin \{i,j,n,\sigma(n)\}$,那么
        % $$
        % \tau\sigma\tau^{-1}\sigma(l)=\tau\sigma\tau^{-1}(\sigma(l))=\tau\sigma(\sigma(l))=\tau(\sigma^2(l)),
        % $$
        % 故$\tau(\sigma^2(l))=l$, 从而$\sigma^2(l)=\tau^{-1}(l)\in \{i,j,n,\sigma(n)\}$.
        
        % 如果有$\sigma(l)\in \{i,j,n,\sigma(n)\}$而$\sigma^2(l)=\sigma(\sigma(l))\notin \{i,j,n,\sigma(n)\}$, 

        % 考虑$\sigma^2$, $\forall\ l\in\{i,j,n,\sigma(n)\}$, 如果有$\sigma^2(l)=\sigma(\sigma(l))\notin \{i,j,n,\sigma(n)\}$,那么
        % $$
        % \tau\sigma\tau^{-1}\sigma(\sigma(l))=\tau\sigma\tau^{-1}(\sigma^2(l))=
        % $$

        \begin{note}
            做法过于多样.
        \end{note}
        B3-4)

        由于B3-3), $\sigma:\{i,j\}\to\{i,j\}$, 由于$n\geqslant 6$, 可知一定存在$j'\in\{1,2,\cdots,n\}\backslash\{i,j,n,\sigma(n)\}$,
        同样有$\sigma:\{i,j'\}\to\{i,j'\}$, 可知$\sigma(i)=i,\forall i\notin\{n,\sigma\}$, 故$\sigma=\{n,\sigma(n)\}$, 从而矛盾.

        综上, $\mathfrak{A_n}$是单群.


    \end{proof}

    \begin{problem}[B4)]
    \end{problem}
    \begin{proof}
        考虑$G \cap \mathfrak{A}_n$, 则$\forall\ g\in G\cap \mathfrak{A}_n, \forall\ \sigma\in \mathfrak{A}_n$, 有
        $$
        \sigma g \sigma^{-1}\in G,\quad \sigma g \sigma^{-1}\in \mathfrak{A}_n,
        $$
        则$G \cap \mathfrak{A}_n$是$\mathfrak{A}_n$的一个正规子群.
        由$\mathfrak{A}_n$的单群性质, 知$G \cap \mathfrak{A}_n=\{1\}$或者$\mathfrak{A}_n$.

        如果$G \cap \mathfrak{A}_n=\mathfrak{A}_n$, 则$\mathfrak{A}_n\subset G$, 
        $|G|\geqslant |\mathfrak{A}_n|=\frac{1}{2}n!$.
        又由$|G|\big| |\mathfrak{S}_n|$知, $G=\mathfrak{A}_n$.
        如果$G \cap \mathfrak{A}_n=\{1\}$, 则$G$中所有非单位元元素都是奇置换,
        故$|G|\leqslant 2$, 否则两个奇置换的乘积为偶置换. 由于$G$非平凡, 故$|G|=2$, 
        则$G=\{1,\tau\}$, 其中$\tau$是一个奇置换. $\forall\ \sigma\in \mathfrak{S}_n$, 有
        $$
        \sigma \tau \sigma^{-1}\in G.
        $$
        因为$\tau$是奇置换, 故$\sigma \tau \sigma^{-1}$也是奇置换, 于是$\sigma \tau \sigma^{-1}=\tau$.
        由此可知$\tau$与$\mathfrak{S}_n$中所有元素交换, 故$\tau$在$\mathfrak{S}_n$的中心中.
        由于$n\geqslant 3$, 故$\mathfrak{S}_n$的中心为平凡群, 则$\tau=1$, 矛盾.
        
        综上, $G=\mathfrak{A}_n$.
    \end{proof}

    \begin{problem}[B5)]
    \end{problem}

    \begin{proof}
        首先有
        \begin{align*}
        ((12)(34))((13)(24))=(14)(23),\\
        ((12)(34))((14)(23))=(13)(24),\\
        ((13)(24))((14)(23))=(12)(34).
        \end{align*}
        于是$(12)(34),(13)(24),(14)(23)$在$N$中乘法封闭且阶都为2. 故$N$是$\mathfrak{S}_n$的子群.
        由于$\mathfrak{S}_n$中的共轭作用保持置换的型不变, 而$N$中的非单位元元素都是(2,2)-型置换, 
        故$N$在$\mathfrak{S}_n$的共轭作用下不变, 于是$N$是$\mathfrak{S}_n$的正规子群.
        $N\triangleleft \mathfrak{A}_4$同理可证.

        考虑$\mathfrak{S}_4$在集合
        $$
        X=\left\{\{\{1,2\},\{3,4\}\},\{\{1,3\},\{2,4\}\},\{\{1,4\},\{2,3\}\}\right\}
        $$
        上的作用, 定义$\pi$在$X$上的作用为: $\pi\cdot \{\{a,b\},\{c,d\}\}=\{\{\pi(a),\pi(b)\},\{\pi(c),\pi(d)\}\}$.

        这是对$X$中3个元素的置换, 因此得到一个群同态$\varphi:\mathfrak{S}_4\to \mathfrak{S}_3$.
        由于$\mathfrak{S}_3$能被一个2-循环和一个3-循环生成, 而$\varphi((12))=(\{\{1,4\},\{2,3\}\},\{\{1,3\},\{2,4\}\})$,
        $\varphi((123))=(\{\{1,2\},\{3,4\}\},\{\{1,4\},\{2,3\}\},\{\{1,3\},\{2,4\}\})$, 这两个3-循环能够生成$\mathfrak{S}_3$,
        故$\varphi$是满同态. 
        
        下面求$\ker\varphi$. 容易观察到$\varphi$将$\mathfrak{S}_4$中相同的型映到$\mathfrak{S}_3$中相同的型上.
        因此, 通过计算, $\ker\varphi$中元素的型只能是(1),(2,2). 
        显然, $\varphi$的核中包含单位元和所有(2,2)-型置换, 即$\ker\varphi=N$.

        由第一同构定理, 有$\mathfrak{S}_4/N\simeq \mathfrak{S}_3$.
    \end{proof}

    \begin{note}
        可以定义自然的群同态$\varphi$:
        \begin{align*}
            \mathfrak{S}_3 &\to \mathfrak{S}_4/N\\
            \sigma         &\mapsto \sigma N.
        \end{align*}
        再证明这是一个群同构.
    \end{note}

    \begin{problem}[B6)]
    \end{problem}

    \begin{proof}
        考虑$\mathfrak{S}_n$作用在左陪集集合$\mathfrak{S}_n/H$上, 定义作用为:
        \begin{align*}
        \mathfrak{S}_n\times \mathfrak{S}_n/H &\to \mathfrak{S}_n/H\\
        (\sigma,\tau H)&\mapsto \sigma\cdot \tau H=(\sigma\tau)H, 
        \end{align*}
        这是一个群作用. 由此得到一个群同态$\varphi:\mathfrak{S}_n\to \mathfrak{S}_{|\mathfrak{S}_n/H|}$.
        由于$|\mathfrak{S}_n/H|=\frac{|\mathfrak{S}_n|}{|H|}=\frac{n!}{|H|}=d$, 
        故$\varphi$可以看作$\mathfrak{S}_n\to \mathfrak{S}_d$的一个群同态.
        
        下面求$\ker\varphi$. $\forall\ \sigma\in \ker\varphi$, 有$\sigma\cdot \tau H=\tau H,\forall\ \tau H\in \mathfrak{S}_n/H$,
        则$\sigma\tau H=\tau H$, 即$\tau^{-1}\sigma\tau\in H,\forall\ \tau\in \mathfrak{S}_n$.
        取$\tau=1$, 则$\sigma\in H$.  于是$\ker\varphi\subset H$, 从而$\ker\varphi<H$.

        我们熟知群同态的核为正规子群, 所以由B5)知$\ker\varphi\in \{1,\mathfrak{A}_n,\mathfrak{S}_n\}$.
        又由$H\neq\mathfrak{A}_n,\mathfrak{S}_n$, 所以$\ker\varphi=1$. 从而$\varphi$是单映射, 推出$d\geqslant n$.
    \end{proof}

    \begin{problem}[B7)]
    \end{problem}

    \begin{proof}
        构造地, 定义群同态$\varphi$:
        \begin{align*}
            \mathfrak{S}_n &\to \mathfrak{A}_{n+2}\\
            \sigma         &\mapsto 
            \begin{cases}
                \sigma,\ &\text{如果$\sigma$是偶置换},\\
                \sigma(n+1,n+2),\ &\text{如果$\sigma$是奇置换}.
            \end{cases}
        \end{align*}
        容易验证这的确是一个单的群同态.

        假设存在这样的单同态$\varphi:\mathfrak{S}_n\to\mathfrak{A}_{n+1}$, 
        知$\mathfrak{S}_n$可以视作$\mathfrak{A}_{n+1}$的一个子群, 故$|\mathfrak{S}_n|\big||\mathfrak{A}_{n+1}|$.
        推出$2\mid n+1$, 从而当$n$为偶数时不成立. 当$n=3$时, $|\mathfrak{S}_3|=6, |\mathfrak{A}_{4}|=12$, 
        但是$\mathfrak{A}_{n+1}$没有阶为6的子群, 矛盾.

        记$G=\varphi(\mathfrak{S}_n)$,考虑群同态$\psi$:
        \begin{align*}
            \mathfrak{A}_{n+1}&\to \mathfrak{A}_{n+1}/G\\
            \sigma            &\mapsto \sigma G
        \end{align*}
        由于$|\mathfrak{A}_{n+1}|=\frac{1}{2}(n+1)!,\ |\mathfrak{A}_{n+1}/G|=\frac{n+1}{2}$, 知$\psi$一定不是单同态.
        于是$\ker\psi\neq 1$, 又$\ker\psi$是$\mathfrak{A}_{n+1}$的正规子群, 故$\ker\psi=1$或者$\mathfrak{A}_{n+1}$.
        于是$\ker\psi=\mathfrak{A}_{n+1}$. 由此得到$\mathfrak{A}_{n+1}\subset G$, 矛盾.

        所以不存在这样的单同态$\varphi:\mathfrak{S}_n\to\mathfrak{A}_{n+1}$.
    \end{proof}
\end{document}