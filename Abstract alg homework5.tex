\documentclass[a4paper, 12pt, UTF8, heading=true, scheme=chinese]{ctexart}

\usepackage[a4paper, left=2cm, right=2cm, top=2cm, bottom=2cm]{geometry}
\usepackage{amsmath,amssymb,bm,graphicx,xcolor,tikz,array,booktabs,multicol,multirow,titlesec,hyperref,biblatex,algorithm2e,listings,mathtools,tikz-cd}

\titleformat{\section}
  {\normalfont\Large\bfseries\raggedright}
  {}
  {0pt}
  {}

\newcommand{\R}{\mathbb{R}}
\newcommand{\Z}{\mathbb{Z}}
\newcommand{\N}{\mathbb{N}}
\newcommand{\Q}{\mathbb{Q}}
\newcommand{\C}{\mathbb{C}}
\newcommand{\st}{\text{s.t.}}

\newenvironment{solution}{\par\noindent\textbf{解:}\par}{\hfill$\square$\par}
\newenvironment{proof}{\par\noindent\textbf{证明:}}{\hfill$\square$\par}
\newenvironment{problem}[1][]{\par\noindent\textmd{#1}}{}
\newenvironment{note}{\par\noindent\textbf{注:}}{\par}
\newenvironment{tip}{\par\noindent\textbf{提示:}}{\par}

\linespread{1.5}

\title{\textbf{群 与 Galois 理论 \\ 作业5}}
\author{陈宏泰 \\ 2024011131 \\ 清华大学数学科学系 \\ \texttt{cht24@mails.tsinghua.edu.cn}}
\date{\today}

\begin{document}

\maketitle 

\tableofcontents
\newpage

\section{A. 分式域的推广: 局部化}

    在此问题中,字母 $A$ 表示是某个给定的交换环,

    \begin{problem}[A1)]
        给定子集 $S \subset A$ ,如果
        \begin{itemize}
            \item $1 \in S ;$
            \item 对任意的 $s_1, s_2 \in S$ ,有 $s_1 \cdot s_2 \in S$ .
        \end{itemize}
        我们就称 $S$ 是{\heiti 乘性子集}.证明,以下两个集合是乘性子集:$\left\{1, f, f^2, \cdots\right\}$ ,其中,$f \in A ; A-\mathfrak{p}$ ,其中, $\mathfrak{p}$ 是素理想(特别的,如果 $A$ 是整环,$A-\{0\}$ 是乘性子集) .
    \end{problem}

    \begin{proof}
        设$S_1=\left\{1, f, f^2, \cdots\right\}$, 则$1 \in S_1$显然成立. 对于任意的$s_1, s_2 \in S_1$, 存在非负整数$m, n$, 使得 $s_1=f^m, s_2=f^n$. 因此, $s_1 \cdot s_2=f^{m+n} \in S_1$. 所以, $S_1$是乘性子集.

        设$S_2=A-\mathfrak{p}$, 则$1 \in S_2$显然成立, 否则$\mathfrak{p}=A$, 这与$\mathfrak{p}$为素理想矛盾. 对于任意的$s_1, s_2 \in S_2$, 如果$s_1 \cdot s_2 \notin S_2$, 则$s_1 \cdot s_2 \in \mathfrak{p}$. 由于$\mathfrak{p}$是素理想, 所以$s_1\in \mathfrak{p}$或$s_2 \in \mathfrak{p}$, 这与$s_1, s_2 \in S_2$矛盾. 因此, $s_1 \cdot s_2 \in S_2$. 所以, $S_2$是乘性子集.
    \end{proof}

    \begin{problem}[A2)]
        我们在 $A \times S$ 上定义等价关系:$(a, s) \sim\left(a^{\prime}, s^{\prime}\right)$ 指的是存在 $t \in S$ ,使得 $a s^{\prime} \cdot t=a^{\prime} s \cdot t$ .证明,以上给出了 $A \times S$ 上的一个等价关系.
        令 $A_S=A \times S / \sim$ ,我们用 $\frac{a}{s}$ 表示 $(a, s)$ 所在的等价类.证明,对任意的 $s^{\prime} \in S$ ,我们有 $\frac{s^{\prime} a}{s^{\prime} s}=\frac{a}{s}$ .
    \end{problem}

    \begin{proof}
        设$(a, s), (a', s'), (a'', s'') \in A \times S$.

        (自反性) 取$t=1 \in S$, 则$as \cdot t = as \cdot t$, 所以$(a, s) \sim (a, s)$.

        (对称性) 如果$(a, s) \sim (a', s')$, 则存在$t \in S$, 使得$as' \cdot t = a's \cdot t$. 自然有$a's \cdot t = as' \cdot t$,因此, $(a', s') \sim (a, s)$.

        (传递性) 如果$(a, s) \sim (a', s')$且$(a', s') \sim (a'', s'')$, 则存在$t_1, t_2 \in S$, 使得$as' \cdot t_1 = a's \cdot t_1$且$a's'' \cdot t_2 = a''s' \cdot t_2$. 
        对前式两边同乘$s'' \cdot t_2$, 对后式两边同乘$s \cdot t_1$, 由$A$为交换环有
        $$
        as' s'' \cdot (t_1 t_2) = a's s'' \cdot (t_1 t_2), \quad a's'' s \cdot (t_1 t_2) = a''s' s \cdot (t_1 t_2).
        $$
        从而有
        $$
        as''\cdot(s' t_1 t_2) = a'ss'' \cdot (t_1 t_2)=a's''s\cdot(t_1t_2)=a''s\cdot(s't_1t_2)
        $$
        由$S$为乘性子群, 知$s't_1t_2\in S$, 从而有$(a,s)\sim(a'',s'')$.

        综上, $\sim$是$A\times S$上的等价关系.

        下面证明对于任意的$s' \in S$, 有$\frac{s'a}{s's} = \frac{a}{s}$.
        只需证明$(s'a, s's) \sim (a, s)$即可.
        对于任意的$s' \in S$, 
        $$
        (s'a)s\cdot1=s'as=as's\cdot1,
        $$
        $1\in S$, 所以$(s'a,s's)\sim(a,s)$.
    \end{proof}

    \begin{note}
        要求$s'\in S$是保证$s's\in S$. 因此就算$s'\notin S$, 但只要$s's\in S$也能保证有$(s'a,s's)\sim(a,s)$.
    \end{note}


    \begin{problem}[A3)]
        我们在 $A_S$ 上定义如下的加法和乘法:
        $$
        \frac{a}{s}+\frac{b}{t}=\frac{a t+b s}{s t}, \quad \frac{a}{s} \cdot \frac{b}{t}=\frac{a b}{s t} .
        $$
        通过验证以上是良定义的来证明,$A_S$ 在以上运算下成为一个环并指出它的乘法和加法单位元.进一步,我们还有自然的环同态:
        $$
        \iota: A \rightarrow A_S, a \mapsto \frac{a}{1} .
        $$
        我们把 $A_S$ 称作是 $A$ 对乘性子集 $S$ 的{\heiti 局部化}.
    \end{problem}

    \begin{proof}
        设$\frac{a}{s} = \frac{a'}{s'}$且$\frac{b}{t} = \frac{b'}{t'}$.

        (加法良定义) 则存在$t_1, t_2 \in S$, 使得$as' \cdot t_1 = a's \cdot t_1$且$bt' \cdot t_2 = b't \cdot t_2$. 对前式两边同乘$tt't_2$, 后式两边同乘$ss't_1$, 得
        $$
        ats't'\cdot(t_1 t_2) = a't'st\cdot (t_1 t_2), \quad bts't' \cdot (t_1 t_2) = b's'st \cdot (t_1 t_2).
        $$
        将这两式相加, 即有
        $$
        (at + bs)s't' \cdot (t_1 t_2) = (a't' + b's')st \cdot (t_1 t_2).
        $$
        又由$S$为乘性子集, 知$t_1t_2\in S$. 因此, $(at + bs, st)\sim(a't'+b's', s't')$说明$\frac{at+bs}{st}=\frac{a't'+b's'}{s't'}$, 从而加法良定义. 

        (乘法良定义) 则存在$t_1, t_2 \in S$, 使得$as' \cdot t_1 = a's \cdot t_1$且$bt' \cdot t_2 = b't \cdot t_2$. 将上面两式相乘, 有
        $$
        abs't'\cdot(t_1t_2)=a'b'st\cdot(t_1t_2).
        $$
        又由$S$为乘性子集, 知$t_1t_2\in S$. 因此, $(ab, st)\sim(a'b', s't')$说明$\frac{ab}{st}=\frac{a'b'}{s't'}$, 从而乘法良定义.

        综上, $A_S$在以上运算下良定义. 下面验证环的公理.
        
        (加法交换律) 对于任意的$\frac{a}{s}, \frac{b}{t} \in A_S$, 有
        $$\frac{a}{s} + \frac{b}{t} = \frac{at + bs}{st} = \frac{bs+ at}{ts} = \frac{b}{t} + \frac{a}{s}.$$

        (加法结合律) 对于任意的$\frac{a}{s}, \frac{b}{t}, \frac{c}{u} \in A_S$, 有
        $$(\frac{a}{s} + \frac{b}{t}) + \frac{c}{u} = \frac{(at + bs)u + cst}{stu} = \frac{au t + bu s + cst}{stu} = \frac{a}{s} + (\frac{b}{t} + \frac{c}{u}).$$

        (加法单位元) 对于任意的$\frac{a}{s} \in A_S$, 有
        $$\frac{a}{s} + \frac{0}{1} = \frac{a \cdot 1 + 0 \cdot s}{s \cdot 1} = \frac{a}{s}.$$

        (加法逆元) 对于任意的$\frac{a}{s} \in A_S$, 有
        $$\frac{a}{s} + \frac{-a}{s} = \frac{as - as}{ss} = \frac{0}{1}.$$

        (乘法交换律) 对于任意的$\frac{a}{s}, \frac{b}{t} \in A_S$, 有
        $$\frac{a}{s} \cdot \frac{b}{t} = \frac{ab}{st} = \frac{ba}{ts} = \frac{b}{t} \cdot \frac{a}{s}.$$

        (乘法结合律) 对于任意的$\frac{a}{s}, \frac{b}{t}, \frac{c}{u} \in A_S$, 有
        $$(\frac{a}{s} \cdot \frac{b}{t}) \cdot \frac{c}{u} = \frac{ab}{st} \cdot \frac{c}{u} = \frac{abc}{stu} = \frac{a}{s} \cdot (\frac{b}{t} \cdot \frac{c}{u}).$$

        (乘法单位元) 对于任意的$\frac{a}{s} \in A_S$, 有
        $$\frac{a}{s} \cdot \frac{1}{1} = \frac{a \cdot 1}{s \cdot 1} = \frac{a}{s}.$$

        (分配律) 对于任意的$\frac{a}{s}, \frac{b}{t}, \frac{c}{u} \in A_S$, 有
        $$\frac{a}{s} \cdot (\frac{b}{t} + \frac{c}{u}) = \frac{a}{s} \cdot \frac{bu + ct}{tu} = \frac{a(bu + ct)}{stu} = \frac{abu}{stu} + \frac{act}{stu} = \frac{a}{s} \cdot \frac{b}{t} + \frac{a}{s} \cdot \frac{c}{u}.$$

        综上, $A_S$在以上运算下成为一个环. 加法单位元为$\frac{0}{1}$, 乘法单位元为$\frac{1}{1}$.

        下面验证$\iota: A \rightarrow A_S, a \mapsto \frac{a}{1}$为环同态.
        对于任意的$a, b \in A$, 有
        $$\iota(a + b) = \frac{a + b}{1} = \frac{a \cdot 1 + b \cdot 1}{1 \cdot 1} = \frac{a}{1} + \frac{b}{1} = \iota(a) + \iota(b),$$
        $$\iota(ab) = \frac{ab}{1} = \frac{a \cdot b}{1 \cdot 1} = \frac{a}{1} \cdot \frac{b}{1} = \iota(a) \cdot \iota(b).$$
        因此, $\iota$为环同态.
    \end{proof}

    \begin{problem}[A4)]
        令 $S_0=\{a \in A \mid a b=0 \Leftrightarrow b=0\}$ .证明,$S_0$ 是乘性子集.我们称 $A_{S_0}$ 为 $A$ 的{\heiti 全分式环}.进一步证明 $\iota: A \rightarrow A_{S_0}$ 是单射并且此时 $\frac{a}{s}=\frac{a^{\prime}}{s^{\prime}}$ 当且仅当 $a s^{\prime}=a^{\prime} s$ .
    \end{problem}

    \begin{proof}
        设$S_0=\{a \in A \mid ab=0 \Leftrightarrow b=0\}$.

        (乘性子集) 显然$1 \in S_0$. 对于任意的$s_1, s_2 \in S_0$, 如果$s_1 s_2 b = 0$, 则$s_1 (s_2 b) = 0$. 由于$s_1 \in S_0$, 所以$s_2 b = 0$. 又由于$s_2 \in S_0$, 所以$b = 0$. 因此, $s_1 s_2 \in S_0$. 所以, $S_0$是乘性子集.

        (单射) 设$\iota(a) = \iota(a')$, 则$\frac{a}{1} = \frac{a'}{1}$. 根据等价关系的定义, 存在$t \in S_0$, 使得$a \cdot 1 \cdot t = a' \cdot 1 \cdot t$, 即$at = a't$. 因为$t \in S_0$, 所以$a = a'$. 因此, $\iota$为单射.

        (充要条件)如果$\frac{a}{s} = \frac{a'}{s'}$, 则存在$t \in S_0$, 使得$as' \cdot t = a's \cdot t$. 因为$t \in S_0$, 所以$as' = a's$. 反之, 如果$as' = a's$, 则对于任意的$t \in S_0$, 有$as' \cdot t = a's \cdot t$. 因此, $\frac{a}{s} = \frac{a'}{s'}$.

        综上, $\iota: A \rightarrow A_{S_0}$ 是单射并且$\frac{a}{s}=\frac{a^{\prime}}{s^{\prime}}$当且仅当$as' = a's$.
    \end{proof}

    \begin{problem}[A5)]
        给定乘性子集 $S \subset A$ .证明, $\operatorname{Ker}(\iota)=\{a \in A \mid$ 存在 $s \in S$ ,使得 $a s=0\}$ .进一步证明,$\iota$ 为单射当且仅当 $S \subset S_0$ .
    \end{problem}

    \begin{proof}
        设$K = \{a \in A \mid$ 存在 $s \in S$ ,使得 $a s=0\}$.

        ($\text{Ker}(\iota$)的刻画) 如果$a \in \operatorname{Ker}(\iota)$, 则$\iota(a) = \frac{a}{1} = \frac{0}{1}$. 根据等价关系的定义, 存在$t \in S$, 使得$a \cdot 1 \cdot t = 0 \cdot 1 \cdot t$, 即$at = 0$. 因此, $a \in K$. 反之, 如果$a \in K$, 则存在$s \in S$, 使得$as = 0$. 对于任意的$t \in S$, 有$as \cdot t = 0 \cdot t$, 即$a \cdot 1 \cdot (st) = 0 \cdot 1 \cdot (st)$. 因为$st \in S$, 所以$\frac{a}{1} = \frac{0}{1}$, 即$a \in \operatorname{Ker}(\iota)$. 因此, $\operatorname{Ker}(\iota) = K$.

        (单射条件) 如果$\iota$为单射, 则$\operatorname{Ker}(\iota) = \{0\}$. 根据上面的等式, 对于任意的$s \in S$, 如果$as = 0$, 则$a = 0$. 因此, $s \in S_0$. 所以, $S \subset S_0$. 反之, 如果$S \subset S_0$, 则对于任意的$a \in \operatorname{Ker}(\iota)$, 存在$s \in S$, 使得$as = 0$. 因为$s \in S_0$, 所以$a = 0$. 因此, $\operatorname{Ker}(\iota) = \{0\}$, 即$\iota$为单射.

        综上, $\operatorname{Ker}(\iota)=\{a \in A \mid$ 存在 $s \in S$ ,使得 $a s=0\}$ 并且$\iota$为单射当且仅当$S \subset S_0$.
    \end{proof}

    \begin{problem}[A6)]
        (局部化的泛性质) $A, S, A_S$ 和 $\iota: A \rightarrow A_S$ 如上述.试验证,$\iota(S) \subset\left(A_S\right)^{\times}$.

        $$
        \begin{tikzcd}
        A \arrow[r, "\varphi"] \arrow[d, "\iota"] & B \\
        A_S \arrow[ru, dashed, "\psi"'] &
        \end{tikzcd}
        $$

        证明,对任意的环 $B$ 和环同态 $\varphi: A \rightarrow B$ ,如果 $\varphi(S) \subset B^{\times}$,则存在唯一的环同态 $\psi: A_S \rightarrow B$ ,使得 $\psi \circ \iota=\varphi$ .
    \end{problem}

    \begin{proof}
        (包含关系) 对于任意的$s \in S$, 有
        $$\iota(s) = \frac{s}{1}.$$
        设$\frac{1}{s}$为$\frac{s}{1}$的乘法逆元, 则有
        $$\frac{s}{1} \cdot \frac{1}{s} = \frac{s \cdot 1}{1 \cdot s} = \frac{1}{1}.$$
        因此, $\iota(s) \in (A_S)^{\times}$. 所以, $\iota(S) \subset (A_S)^{\times}$.

        (存在性) 定义映射$\psi: A_S \rightarrow B$, 对于任意的$\frac{a}{s} \in A_S$, 令
        $$\psi(\frac{a}{s}) = \varphi(a) \cdot \varphi(s)^{-1}.$$
        下面验证$\psi$为环同态且$\psi \circ \iota = \varphi$.

        (良定义) 如果$\frac{a}{s} = \frac{a'}{s'}$, 则存在$t \in S$, 使得$as' \cdot t = a's \cdot t$. 应用$\varphi$, 有
        $$\varphi(a) \varphi(s') \varphi(t) = \varphi(a') \varphi(s) \varphi(t).$$
        由于$\varphi(t),\varphi(s),\varphi(s') \in B^{\times}$, 所以
        $$\psi(\frac{a}{s})=\varphi(a) \varphi(s)^{-1} = \varphi(a') \varphi(s)^{-1}=\psi(\frac{a'}{s'}).$$
        因此, $\psi$良定义.

        (环同态) 对于任意的$\frac{a}{s}, \frac{b}{t} \in A_S$, 有
        \begin{align*}
        \psi(\frac{a}{s} + \frac{b}{t}) &= \psi(\frac{at + bs}{st}) = \varphi(at + bs) \cdot \varphi(st)^{-1} \\
        &= (\varphi(a)\varphi(t) + \varphi(b)\varphi(s)) \cdot (\varphi(s)\varphi(t))^{-1} \\
        &= \varphi(a)\varphi(s)^{-1} + \varphi(b)\varphi(t)^{-1} = \psi(\frac{a}{s}) + \psi(\frac{b}{t}),
        \end{align*}
        \begin{align*}
        \psi(\frac{a}{s} \cdot \frac{b}{t}) &= \psi(\frac{ab}{st}) = \varphi(ab) \cdot \varphi(st)^{-1} \\
        &= (\varphi(a)\varphi(b)) \cdot (\varphi(s)\varphi(t))^{-1} \\
        &= \varphi(a)\varphi(s)^{-1} \cdot \varphi(b)\varphi(t)^{-1} = \psi(\frac{a}{s}) \cdot \psi(\frac{b}{t}).
        \end{align*}
        因此, $\psi$为环同态.

        对于任意的$a \in A$, 有
        $$\psi(\iota(a)) = \psi(\frac{a}{1}) = \varphi(a) \cdot \varphi(1)^{-1} = \varphi(a) \cdot 1 = \varphi(a).$$
        因此, $\psi \circ \iota = \varphi$.

        (唯一性) 如果存在环同态$\psi': A_S \rightarrow B$, 使得$\psi' \circ \iota = \varphi$, 则对于任意的$\frac{a}{s} \in A_S$, 有
        $$\psi'(\frac{a}{s}) = \psi'(\iota(a) \cdot \iota(s)^{-1}) = \psi'(\iota(a)) \cdot \psi'(\iota(s))^{-1} = \varphi(a) \cdot \varphi(s)^{-1} = \psi(\frac{a}{s}).$$
        因此, $\psi' = \psi$.
    \end{proof}

    \begin{problem}[A7)]
        $A, S, A_S$ 和 $\iota: A \rightarrow A_S$ 如上述, $\widehat{S}=\{a \in A \mid$ 存在 $b \in A$, 使得 $a b \in S\}$. 证明, $\widehat{S}=\iota^{-1}\left(\left(A_S\right)^{\times}\right)$. 进一步证明环同构 $A_S \xrightarrow{\simeq} A_{\widehat{S}}$, 其中, $\frac{a}{1}$ 的像是 $\frac{a}{1}$.
    \end{problem}

    \begin{proof}
        设$\widehat{S}=\{a \in A \mid$ 存在 $b \in A$ ,使得 $a b \in S\}$.

        ($\widehat{S}$的刻画) 如果$a \in \widehat{S}$, 则存在$b \in A$, 使得$ab \in S$. 设$\iota(a) = \frac{a}{1}$, 则
        $$\iota(a) \cdot \iota(b) = \frac{a}{1} \cdot \frac{b}{1} = \frac{ab}{1}.$$
        因为$ab \in S$, 所以$\frac{ab}{1} \in (A_S)^{\times}$. 因此, 
        $$
        \frac{1}{a}\cdot (\frac{1}{b}(\frac{1}{ab})^{-1})=\frac{1}{1},
        $$
        所以$\iota(a) \in (A_S)^{\times}$, 即$a \in \iota^{-1}((A_S)^{\times})$. 反之, 如果$a \in \iota^{-1}((A_S)^{\times})$, 则$\iota(a) = \frac{a}{1} \in (A_S)^{\times}$. 设$\frac{c}{d}$为$\frac{a}{1}$的乘法逆元, 则有
        $$\frac{a}{1} \cdot \frac{c}{d} = \frac{ac}{d} = \frac{1}{1}.$$
        根据等价关系的定义, 存在$t \in S$, 使得$ac \cdot t = d \cdot t$. 因为$S$为乘性子集, 所以$dt \in S$. 因此, $a(c t) = dt \in S$, 即$a \in \widehat{S}$. 综上, $\widehat{S} = \iota^{-1}((A_S)^{\times})$.

        再证明环同构$A_S \xrightarrow{\simeq} A_{\widehat{S}}$.
        定义映射$\phi: A_S \rightarrow A_{\widehat{S}}$, 对于任意的$\frac{a}{s} \in A_S$, 令
        $$\phi(\frac{a}{s}) = \frac{a}{s},$$
        其中右边的分式表示在$A_{\widehat{S}}$中. 下面验证$\phi$为环同构.

        (良定义) 如果$\frac{a}{s} = \frac{a'}{s'}$, 则存在$t \in S$, 使得$as' \cdot t = a's \cdot t$. 因为$S \subset \widehat{S}$, 所以$t \in \widehat{S}$. 因此, $(a, s) \sim (a', s')$在$A_{\widehat{S}}$中亦成立, 即$\frac{a}{s} = \frac{a'}{s'}$在$A_{\widehat{S}}$中成立. 所以, $\phi$良定义.

        (环同态) 对于任意的$\frac{a}{s}, \frac{b}{t} \in A_S$, 有
        \begin{align*}
        \phi(\frac{a}{s} + \frac{b}{t}) &= \phi(\frac{at + bs}{st}) = \frac{at + bs}{st} = \frac{a}{s} + \frac{b}{t}, \\
        \phi(\frac{a}{s} \cdot \frac{b}{t}) &= \phi(\frac{ab}{st}) = \frac{ab}{st} = \frac{a}{s} \cdot \frac{b}{t}.
        \end{align*}
        因此, $\phi$为环同态.

        (双射性) 如果$\phi(\frac{a}{s}) = \phi(\frac{a'}{s'})$, 则$\frac{a}{s} = \frac{a'}{s'}$在$A_{\widehat{S}}$中成立. 根据等价关系的定义, 存在$t \in \widehat{S}$, 使得$as' \cdot t = a's \cdot t$. 由$t \in \widehat{S}$, 存在$c \in A$, 使得$tc \in S$. 因此, $as' \cdot (tc) = a's \cdot (tc)$, 即$(as', ss') \sim (a's', ss')$在$A_S$中成立, 即$\frac{a}{s} = \frac{a'}{s'}$在$A_S$中成立. 所以, $\phi$为单射.
        对于任意的$\frac{a}{s} \in A_{\widehat{S}}$, 因为$s \in \widehat{S}$, 存在$c \in A$, 使得$sc \in S$. 因此, 有
        $$\phi(\frac{ac}{sc}) = \frac{ac}{sc} = \frac{a}{s}.$$
        所以, $\phi$为满射.
        综上, $\phi$为环同构.
    \end{proof}

    \begin{problem}[A8)]
        $A$ 和 $B$ 是交换环,$\varphi: A \rightarrow B$ 是环同态,$S \subset A$ 和 $T \subset B$ 是乘性子集并且 $\varphi(S) \subset T$ .证明,存在唯一的环同态 $\psi: A_S \rightarrow B_T$ ,使得如下图表交换:
        $$        \begin{tikzcd}
        A \arrow[r, "\varphi"] \arrow[d, "\iota"] & B \arrow[d, "\iota"] \\
        A_S \arrow[r, dashed, "\psi"'] & B_T
        \end{tikzcd}$$ 
    \end{problem}

    \begin{proof}
        设$A, B, S, T, \varphi$如题设. 定义映射$\psi: A_S \rightarrow B_T$, 对于任意的$\frac{a}{s} \in A_S$, 令
        $$\psi(\frac{a}{s}) = \frac{\varphi(a)}{\varphi(s)}.$$
        下面验证$\psi$为环同态且图表交换.

        (良定义) 如果$\frac{a}{s} = \frac{a'}{s'}$, 则存在$t \in S$, 使得$as' \cdot t = a's \cdot t$. 应用$\varphi$, 有
        $$\varphi(a) \varphi(s') \varphi(t) = \varphi(a') \varphi(s) \varphi(t).$$
        由于$\varphi(t) \in T$, 所以
        $$\psi(\frac{a}{s})=\frac{\varphi(a)}{\varphi(s)} = \frac{\varphi(a')}{\varphi(s')}=\psi(\frac{a'}{s'}).$$
        因此, $\psi$良定义.

        (环同态) 对于任意的$\frac{a}{s}, \frac{b}{t} \in A_S$, 有
        \begin{align*}
        \psi(\frac{a}{s} + \frac{b}{t}) &= \psi(\frac{at + bs}{st}) = \frac{\varphi(at + bs)}{\varphi(st)} \\
        &= \frac{\varphi(a)\varphi(t) + \varphi(b)\varphi(s)}{\varphi(s)\varphi(t)} \\
        &= \frac{\varphi(a)}{\varphi(s)} + \frac{\varphi(b)}{\varphi(t)} = \psi(\frac{a}{s}) + \psi(\frac{b}{t}),
        \end{align*}
        \begin{align*}
        \psi(\frac{a}{s} \cdot \frac{b}{t}) &= \psi(\frac{ab}{st}) = \frac{\varphi(ab)}{\varphi(st)} \\
        &= \frac{\varphi(a)\varphi(b)}{\varphi(s)\varphi(t)} \\
        &= \frac{\varphi(a)}{\varphi(s)} \cdot \frac{\varphi(b)}{\varphi(t)} = \psi(\frac{a}{s}) \cdot \psi(\frac{b}{t}).
        \end{align*}
        因此, $\psi$为环同态.
        对于任意的$a \in A$, 有
        $$\psi(\iota_A(a)) = \psi(\frac{a}{1}) = \frac{\varphi(a)}{\varphi(1)} = \frac{\varphi(a)}{1} = \iota_B(\varphi(a)).$$
        因此, 图表交换.
    \end{proof}


    \begin{problem}[A9)]
        (理想与局部化) $I \subset A$ 是理想,令 $I_S$ 为 $\iota(I)$ 在 $A_S$ 中生成的理想.
        \begin{itemize}
            \item 证明,$I_S=\left\{\left.\frac{a}{s} \right\rvert\, a \in I, s \in S\right\}$ .进一步证明,$I_S=A_S$ 当且仅当 $S \cap I \neq \emptyset$ .
            \item $J \subset A_S$ 是理想, 证明, $\left(\iota^{-1}(J)\right)_S=J$.
        \end{itemize}
    \end{problem}

    \begin{proof}
        设$I \subset A$为理想, $I_S$为$\iota(I)$在$A_S$中生成的理想.

        (理想的刻画) 设$K = \{\frac{a}{s} \mid a \in I, s \in S\}$.

        先证明$I_S \subset K$. 对于任意的$a \in I$, 有$\iota(a) = \frac{a}{1} \in I_S$. 对于任意的$\frac{b}{t} \in A_S$, 有
        $$\frac{b}{t} \cdot \frac{a}{1} = \frac{ba}{t}.$$
        因为$I$为理想, 所以$ba \in I$. 因此, $\frac{ba}{t} \in K$. 由理想的定义, 知$I_S \subset K$.

        再证明$K \subset I_S$. 对于任意的$\frac{a}{s} \in K$, 有$a \in I$. 因为$\iota(a) = \frac{a}{1} \in I_S$, 所以
        $$\frac{a}{s} = \frac{a}{1} \cdot \frac{1}{s} \in I_S.$$
        因此, $K \subset I_S$.

        综上, $I_S = K = \{\frac{a}{s} \mid a \in I, s \in S\}$.

        进一步证明, $I_S=A_S$当且仅当$S \cap I \neq \emptyset$.

        如果$I_S = A_S$, 则$1_{A_S} = \frac{1}{1} \in I_S$. 根据上面的等式, 存在$a \in I$, $s \in S$, 使得$\frac{a}{s} = \frac{1}{1}$. 根据等价关系的定义, 存在$t \in S$, 使得$a \cdot 1 \cdot t = 1 \cdot s \cdot t$, 即$at = st$. 因为$t, s \in S$, 所以$st \in S$. 由因为$a\in I$, 所以$at\in I$. 因此, $at = st \in S \cap I$. 反之, 如果$S \cap I \neq \emptyset$, 则存在$s \in S \cap I$. 对于任意的$\frac{a}{t} \in A_S$, 有
        $$\frac{a}{t} = \frac{a s}{t s}.$$
        因为$s \in I$, 所以$as \in I$. 因此, $\frac{as}{ts} \in I_S$. 所以, $A_S \subset I_S$. 由理想的定义, 知$I_S \subset A_S$, 从而$I_S=A_S$.

        综上, $I_S = A_S$当且仅当$S \cap I \neq \emptyset$.

        设$J \subset A_S$为理想. 下面证明$(\iota^{-1}(J))_S = J$.

        先证明$(\iota^{-1}(J))_S \subset J$. 对于任意的$\frac{a}{s} \in (\iota^{-1}(J))_S$, 有$a \in \iota^{-1}(J)$, 即$\iota(a) = \frac{a}{1} \in J$. 因为$J$为理想, 所以
        $$\frac{a}{s} = \frac{a}{1} \cdot \frac{1}{s} \in J.$$
        因此, $(\iota^{-1}(J))_S \subset J$.

        再证明$J \subset (\iota^{-1}(J))_S$. 对于任意的$\frac{a}{s} \in J$, 有
        $$
        \iota(a) = \frac{a}{1}= \frac{a}{s}\cdot \frac{s}{1} \in J,
        $$
        即$a \in \iota^{-1}(J)$. 因此,
        $$\frac{a}{s} \in (\iota^{-1}(J))_S.$$
        所以, $J \subset (\iota^{-1}(J))_S$

        综上, $(\iota^{-1}(J))_S = J$.
    \end{proof}

    \begin{problem}[A10)]
        (素理想与局部化)我们证明 $A_S$ 中的素理想与 $A$ 中与 $S$ 不交的素理想一一对应.
        \begin{itemize}
            \item $\mathfrak{p} \subset A$是素理想并且$\mathfrak{p} \cap S=\emptyset$ , 证明, $\mathfrak{p}_S$ 为 $A_S$ 中的素理想.
            \item $\mathfrak{q} \subset A_S$ 是素理想,证明, $\iota^{-1} \mathfrak{q}$ 是 $A$ 中唯一满足 $\mathfrak{p}_S=\mathfrak{q}$ 的素理想.
        \end{itemize}
    \end{problem}

    \begin{proof}
        设$\mathfrak{p} \subset A$为素理想且$\mathfrak{p} \cap S = \emptyset$. 由A9)知, $\mathfrak{p}_S$为$\iota(\mathfrak{p})$在$A_S$中生成的理想且$\mathfrak{p}_S = \{\frac{a}{s} \mid a \in \mathfrak{p}, s \in S\}$, 下面证明$\mathfrak{p}_S$为素理想.

        对于任意的$\frac{a}{s}, \frac{b}{t} \in A_S$, 如果$\frac{a}{s} \cdot \frac{b}{t} = \frac{ab}{st} \in \mathfrak{p}_S$, 则存在$u \in S$, 使得$ab \cdot u \in \mathfrak{p}$ ($\frac{ab}{st}=\frac{abu}{stu}\in \mathfrak{p}_S$). 
        因为$\mathfrak{p}$为素理想, 并且$u\notin \mathfrak{p}$ ($u\in S$且$\mathfrak{p}\cap S=\emptyset$), 所以$ab \in \mathfrak{p}$. 所以$a \in \mathfrak{p}$或$b \in \mathfrak{p}$. 因此, $\frac{a}{s} \in \mathfrak{p}_S$或$\frac{b}{t} \in \mathfrak{p}_S$. 所以, $\mathfrak{p}_S$为素理想.

        设$\mathfrak{q} \subset A_S$为素理想. 下面证明$\iota^{-1}(\mathfrak{q})$为$A$中唯一满足$\mathfrak{p}_S = \mathfrak{q}$的素理想.

        先证明$\iota^{-1}(\mathfrak{q})$为素理想. 对于任意的$a, b \in A$, 如果$ab \in \iota^{-1}(\mathfrak{q})$, 则$\iota(ab) = \frac{ab}{1} \in \mathfrak{q}$. 因为$\mathfrak{q}$为素理想, 所以$\frac{a}{1} \in \mathfrak{q}$或$\frac{b}{1} \in \mathfrak{q}$. 因此, $a \in \iota^{-1}(\mathfrak{q})$或$b \in \iota^{-1}(\mathfrak{q})$. 所以, $\iota^{-1}(\mathfrak{q})$为素理想.
        
        再证明$\mathfrak{p}_S = \mathfrak{q}$的唯一性. 如果存在素理想$\mathfrak{p}' \subset A$, 使得$\mathfrak{p}'_S = \mathfrak{q}$, 则对于任意的$a \in \iota^{-1}(\mathfrak{q})$, 有$\iota(a) = \frac{a}{1} \in \mathfrak{q} = \mathfrak{p}'_S$. 根据上面的等式, 存在$s \in S$, 使得$as \in \mathfrak{p}'$. 因为$s \notin \mathfrak{p}'$(否则$\mathfrak{p}'_S = A_S$), 所以$a \in \mathfrak{p}'$. 因此, $\iota^{-1}(\mathfrak{q}) \subset \mathfrak{p}'$.
        对于任意的$a \in \mathfrak{p}'$, 有$\iota(a) = \frac{a}{1} \in \mathfrak{p}'_S = \mathfrak{q}$. 因此, $a \in \iota^{-1}(\mathfrak{q})$. 所以, $\mathfrak{p}' \subset \iota^{-1}(\mathfrak{q})$. 综上, $\iota^{-1}(\mathfrak{q}) = \mathfrak{p}'$.

        综上, $\iota^{-1}(\mathfrak{q})$为$A$中唯一满足$\mathfrak{p}_S = \mathfrak{q}$的素理想.
    \end{proof}

    \begin{problem}[A11)]
        $\mathfrak{p} \subset A$ 是素理想, $S=A-\mathfrak{p}$, 令 $A_{\mathfrak{p}}=A_S$. 证明, $A_{\mathfrak{p}}$ 是局部环(即只有一个极大理想的环)并确定它的极大理想.
    \end{problem}

    \begin{proof}
        根据A10), $A_{\mathfrak{p}}$中的素理想与$A$中与$S$不交的素理想一一对应. 因为$S = A - \mathfrak{p}$, 所以与$S$不交的素理想只有$\mathfrak{p}$一个. 因此, $A_{\mathfrak{p}}$中只有一个素理想, 即$\mathfrak{p}_S$. 
        又极大理想都是素理想; 而对于$A_\mathfrak{p}$中的任意理想$I\neq A_\mathfrak{p}$, 存在极大理想$\mathfrak{m}\neq A_\mathfrak{p}$使得$I \subset \mathfrak{m}$. 由于$A_\mathfrak{p}$中只有一个极大理想$\mathfrak{p}_S$.

        由此可知, $\mathfrak{p}_S$为$A_{\mathfrak{p}}$的唯一极大理想. 所以, $A_{\mathfrak{p}}$为局部环.
    \end{proof}

    \begin{problem}[A12)]
        (局部化与商可交换)$I \subset A$ 是理想,$S \subset A$ 是乘性子集,$\pi: A \rightarrow A / I$ 是商映射,$\pi(S) \subset A / I$ 也是乘性子集.证明,存在自然的环同构

        $$
        (A / I)_{\pi(S)} \xrightarrow{\simeq} A_S / I_S .
        $$
    \end{problem}

    \begin{proof}
        设$I \subset A$为理想, $S \subset A$为乘性子集, $\pi: A \rightarrow A / I$为商映射, $\pi(S) \subset A / I$亦为乘性子集.

        定义映射$\phi: (A / I)_{\pi(S)} \rightarrow A_S / I_S$, 对于任意的$\frac{a + I}{s + I} \in (A / I)_{\pi(S)}$, 令
        $$\phi(\frac{a + I}{s + I}) = \frac{a}{s} + I_S.$$
        下面验证$\phi$为环同构.

        (良定义) 如果$\frac{a + I}{s + I} = \frac{a' + I}{s' + I}$, 则存在$t + I \in \pi(S)$, 使得$(a + I)(s' + I)(t + I) = (a' + I)(s + I)(t + I)$. 即$as't - a'st \in I$. 因为$t \in S$, 所以$st \in S$. 因此, 有
        $$(as' - a's) t \in I \implies (as' - a's) t = i$$
        对某个$i \in I$. 因为$t \in S$, 所以
        $$\frac{as'}{st} - \frac{a's}{st} = \frac{i}{st} \in I_S.$$
        因此,
        $$\phi(\frac{a + I}{s + I}) = \frac{a}{s} + I_S = \frac{a'}{s'} + I_S = \phi(\frac{a' + I}{s' + I}).$$
        所以, $\phi$良定义.

        (环同态) 对于任意的$\frac{a + I}{s + I}, \frac{b + I}{t + I} \in (A / I)_{\pi(S)}$, 有
        \begin{align*}
        \phi(\frac{a + I}{s + I} + \frac{b + I}{t + I}) &= \phi(\frac{(a + I)(t + I) + (b + I)(s + I)}{(s + I)(t + I)}) \\
        &= \frac{at + bs}{st} + I_S \\
        &= \frac{a}{s} + I_S + \frac{b}{t} + I_S = \phi(\frac{a + I}{s + I}) + \phi(\frac{b + I}{t + I}),
        \end{align*}
        \begin{align*}
        \phi(\frac{a + I}{s + I} \cdot \frac{b + I}{t + I}) &= \phi(\frac{(a + I)(b + I)}{(s + I)(t + I)}) \\
        &= \frac{ab}{st} + I_S \\
        &= (\frac{a}{s} + I_S) \cdot (\frac{b}{t} + I_S) = \phi(\frac{a + I}{s + I}) \cdot \phi(\frac{b + I}{t + I}).
        \end{align*}
        因此, $\phi$为环同态.

        (双射性) 如果$\phi(\frac{a + I}{s + I}) = \phi(\frac{a' + I}{s' + I})$, 则
        $$\frac{a}{s} + I_S = \frac{a'}{s'} + I_S.$$
        根据商理想的定义, 存在$t \in S$, 使得
        $$(as' - a's) t \in I.$$
        因为$t \in S$, 所以$st \in S$. 因此, 有
        $$(as' - a's) t = i$$
        对某个$i \in I$. 由此可知,
        $$(a + I)(s' + I)(t + I) = (a' + I)(s + I)(t + I).$$
        因为$t + I \in \pi(S)$, 所以
        $$\frac{a + I}{s + I} = \frac{a' + I}{s' + I}.$$
        所以, $\phi$为单射.
        对于任意的$\frac{a}{s} + I_S \in A_S / I_S$, 有
        $$\phi(\frac{a + I}{s + I}) = \frac{a}{s} + I_S.$$
        所以, $\phi$为满射.
        综上, $\phi$为环同构.
    \end{proof}

    \begin{problem}[A13)]
        给定 $f \in A, S=\left\{1, f, f^2, \cdots\right\}$ ,记 $A_f=A_S$ .证明,我们有环同构

        $$
        A[X] /(1-f X) \xrightarrow{\simeq} A_f, \quad X \mapsto \frac{1}{f} .
        $$

    \end{problem}

    \begin{proof}
        设$f \in A$, $S = \{1, f, f^2, \ldots\}$, $A_f = A_S$.

        定义映射$\phi: A[X] / (1 - fX) \rightarrow A_f$, 对于任意的$g(X) + (1 - fX) \in A[X] / (1 - fX)$, 令
        $$\phi(g(X) + (1 - fX)) = g(\frac{1}{f}).$$
        下面验证$\phi$为环同构.

        (良定义) 如果$g(X) + (1 - fX) = h(X) + (1 - fX)$, 则$g(X) - h(X) \in (1 - fX)$. 即存在$q(X) \in A[X]$, 使得$g(X) - h(X) = q(X)(1 - fX)$. 因此,
        $$g(\frac{1}{f}) - h(\frac{1}{f}) = q(\frac{1}{f})(1 - f \cdot \frac{1}{f}) = q(\frac{1}{f}) \cdot 0 = 0.$$
        所以, $\phi(g(X) + (1 - fX)) = g(\frac{1}{f}) = h(\frac{1}{f}) = \phi(h(X) + (1 - fX))$. 因此, $\phi$良定义.

        (环同态) 对于任意的$g(X) + (1 - fX), h(X) + (1 - fX) \in A[X] / (1 - fX)$, 有
        \begin{align*}
        \phi((g(X) + (1 - fX)) + (h(X) + (1 - fX))) &= \phi((g(X) + h(X)) + (1 - fX)) \\
        &= (g + h)(\frac{1}{f}) = g(\frac{1}{f}) + h(\frac{1}{f}) \\
        &= \phi(g(X) + (1 - fX)) + \phi(h(X) + (1 - fX)),
        \end{align*}
        \begin{align*}
        \phi((g(X) + (1 - fX)) \cdot (h(X) + (1 - fX))) &= \phi((g(X)h(X)) + (1 - fX)) \\
        &= (gh)(\frac{1}{f}) = g(\frac{1}{f}) \cdot h(\frac{1}{f}) \\
        &= \phi(g(X) + (1 - fX)) \cdot \phi(h(X) + (1 - fX)).
        \end{align*}
        因此, $\phi$为环同态.

        (双射性) 如果$\phi(g(X) + (1 - fX)) = \phi(h(X) + (1 - fX))$, 则$g(\frac{1}{f}) = h(\frac{1}{f})$. 设$d(X) = g(X) - h(X)$, 则$d(\frac{1}{f}) = 0$. 因此, $d(X)$在$X = \frac{1}{f}$处有根, 即$1 - fX$整除$d(X)$. 所以, 存在$q(X) \in A[X]$, 使得$d(X) = q(X)(1 - fX)$. 因此, $g(X) + (1 - fX) = h(X) + (1 - fX)$. 所以, $\phi$为单射.
        对于任意的$\frac{a}{f^n} \in A_f$, 令$g(X) = a X^n \in A[X]$. 则有
        $$\phi(g(X) + (1 - fX)) = g(\frac{1}{f}) = a (\frac{1}{f})^n = \frac{a}{f^n}.$$
        所以, $\phi$为满射.
        综上, $\phi$为环同构.
    \end{proof}

\newpage

\section{B. \texorpdfstring{$\Z[\sqrt{d}]^\times$}. 与Pell方程, \texorpdfstring{$d\neq\square, d>0$}.}

    假设 $d \in \mathbb{Z}$ 不是完全平方数。令
    $$
    \mathbb{Z}[\sqrt{d}]=\{x+y \sqrt{d} \mid x, y \in \mathbb{Z}\}, \quad \mathbb{Q}[\sqrt{d}]=\{x+y \sqrt{d} \mid x, y \in \mathbb{Q}\}
    $$

    \begin{problem}[B1)]
        证明, $\mathbb{Z}[\sqrt{d}]$ 是环而 $\mathbb{Q}[\sqrt{d}]$ 为其分式域。
    \end{problem}

    \begin{proof}
        设$d \in \mathbb{Z}$不是完全平方数. 下面证明$\mathbb{Z}[\sqrt{d}]$为环且$\mathbb{Q}[\sqrt{d}]$为其分式域.

        先证明$\mathbb{Z}[\sqrt{d}]$为环. 对于任意的$x_1 + y_1 \sqrt{d}, x_2 + y_2 \sqrt{d} \in \mathbb{Z}[\sqrt{d}]$, 有
        $$(x_1 + y_1 \sqrt{d}) + (x_2 + y_2 \sqrt{d}) = (x_1 + x_2) + (y_1 + y_2) \sqrt{d} \in \mathbb{Z}[\sqrt{d}],$$
        $$(x_1 + y_1 \sqrt{d}) \cdot (x_2 + y_2 \sqrt{d}) = (x_1 x_2 + y_1 y_2 d) + (x_1 y_2 + x_2 y_1) \sqrt{d} \in \mathbb{Z}[\sqrt{d}].$$
        因此, $\mathbb{Z}[\sqrt{d}]$在加法和乘法下封闭. 显然, 加法和乘法满足交换律和结合律, 且乘法对加法分配. 零元为$0 + 0\sqrt{d}$, 加法逆元为$x - y\sqrt{d}$. 所以, $\mathbb{Z}[\sqrt{d}]$为环.

        再证明$\mathbb{Q}[\sqrt{d}]$为$\mathbb{Z}[\sqrt{d}]$的分式域. 对于任意的$x_1 + y_1 \sqrt{d}, x_2 + y_2 \sqrt{d} \in \mathbb{Q}[\sqrt{d}]$, 有
        $$(x_1 + y_1 \sqrt{d}) + (x_2 + y_2 \sqrt{d}) = (x_1 + x_2) + (y_1 + y_2) \sqrt{d} \in \mathbb{Q}[\sqrt{d}],$$
        $$(x_1 + y_1 \sqrt{d}) \cdot (x_2 + y_2 \sqrt{d}) = (x_1 x_2 + y_1 y_2 d) + (x_1 y_2 + x_2 y_1) \sqrt{d} \in \mathbb{Q}[\sqrt{d}].$$
        因此, $\mathbb{Q}[\sqrt{d}]$在加法和乘法下封闭. 显然, 加法和乘法满足交换律和结合律, 且乘法对加法分配. 零元为$0 + 0\sqrt{d}$, 加法逆元为$x - y\sqrt{d}$. 所以, $\mathbb{Q}[\sqrt{d}]$为环.
        对于任意的$x + y \sqrt{d} \in \mathbb{Q}[\sqrt{d}] - \{0\}$, 令
        $$\frac{1}{x + y \sqrt{d}} = \frac{x - y \sqrt{d}}{x^2 - d y^2}.$$
        因为$x, y \in \mathbb{Q}$且$x^2 - d y^2 \neq 0$(否则$d$为完全平方数), 所以$\frac{1}{x + y \sqrt{d}} \in \mathbb{Q}[\sqrt{d}]$. 因此, $\mathbb{Q}[\sqrt{d}]$为$\mathbb{Z}[\sqrt{d}]$的分式域.
    \end{proof}

    \begin{problem}[B2)]
        证明,如果 $d<0, \mathbb{Z}[\sqrt{d}]$ 是 $\mathbb{C}$ 中的格点(从而是离散的);如果 $d>0, \mathbb{Z}[\sqrt{d}]$ 在 $\mathbb{R}$ 中稠密。
    \end{problem}

    \begin{proof}
        设$d \in \mathbb{Z}$不是完全平方数.

        如果$d < 0$, 则$\sqrt{d} = i \sqrt{|d|}$. 因此, 对于任意的$x + y \sqrt{d} \in \mathbb{Z}[\sqrt{d}]$, 有
        $$x + y \sqrt{d} = x + y i \sqrt{|d|}.$$
        由此可知, $\mathbb{Z}[\sqrt{d}]$为$\mathbb{C}$中的格点. 因为格点是离散的, 所以$\mathbb{Z}[\sqrt{d}]$在$\mathbb{C}$中离散.

        如果$d > 0$, 则对于任意的$x + y \sqrt{d} \in \mathbb{Z}[\sqrt{d}]$, 有
        $$x + y \sqrt{d}.$$
        设$a \in \mathbb{R}$, $\epsilon > 0$. 令$x = \lfloor a \rfloor$, $y = \lfloor \frac{a - x}{\sqrt{d}} \rfloor$. 则有
        $$|a - (x + y \sqrt{d})| < \epsilon.$$
        因此, $\mathbb{Z}[\sqrt{d}]$在$\mathbb{R}$中稠密.
    \end{proof}

    \begin{problem}[B3)]
        对任意的 $z=x+y \sqrt{d} \in \mathbb{Q}[\sqrt{d}]$ ,我们定义 $\bar{z}=x-y \sqrt{d}$(请注意,如果 $d>0$ ,这不是复共轭)。证明,环 $\mathbb{Z}[\sqrt{d}]$ 的自同构群 $\operatorname{Aut}(\mathbb{Z}[\sqrt{d}])$ 恰有 2 个元素。
    \end{problem}

    \begin{proof}
        设$d \in \mathbb{Z}$不是完全平方数. 对于任意的$z = x + y \sqrt{d} \in \mathbb{Q}[\sqrt{d}]$, 定义$\bar{z} = x - y \sqrt{d}$.

        下面证明环$\mathbb{Z}[\sqrt{d}]$的自同构群$\operatorname{Aut}(\mathbb{Z}[\sqrt{d}])$恰有2个元素.

        设$\sigma \in \operatorname{Aut}(\mathbb{Z}[\sqrt{d}])$. 因为$\sigma$为环同构, 所以
        $$\sigma(1) = 1.$$
        设$\sigma(\sqrt{d}) = a + b \sqrt{d}$, 其中$a, b \in \mathbb{Z}$. 则有
        $$\sigma(\sqrt{d})^2 = \sigma(d) = d.$$
        因此,
        $$(a + b \sqrt{d})^2 = a^2 + 2ab \sqrt{d} + b^2 d = d.$$
        由此可知,
        $$a^2 + b^2 d = d,$$
        $$2ab = 0.$$
        如果$b = 0$, 则$a^2 = d$, 与$d$不是完全平方数矛盾. 因此, $a = 0$. 所以, $b^2 d = d$. 因为$d \neq 0$, 所以$b^2 = 1$. 因此, $b = 1$或$b = -1$.

        如果$b = 1$, 则$\sigma(\sqrt{d}) = \sqrt{d}$. 如果$b = -1$, 则$\sigma(\sqrt{d}) = -\sqrt{d}$. 因此, 环$\mathbb{Z}[\sqrt{d}]$的自同构群$\operatorname{Aut}(\mathbb{Z}[\sqrt{d}])$恰有2个元素, 即恒等映射和共轭映射.
    \end{proof}

    \begin{problem}[B4)]
        对任意的 $z \in \mathbb{Q}[\sqrt{d}]$ ,我们定义 $N(z)=z \cdot \bar{z}_{\text {。 }}$ 证明, 对任意的 $a, b \in \mathbb{Q}[\sqrt{d}], N(a \cdot b)=N(a) \cdot N(b)$并且 $N(\mathbb{Z}[\sqrt{d}]) \subset \mathbb{Z}$ 。据此证明: $\mathbb{Z}[\sqrt{d}]^{\times}=\{z \in \mathbb{Z}[\sqrt{d}] \mid N(z)= \pm 1\}$ 。
    \end{problem}

    \begin{proof}
        设$d \in \mathbb{Z}$不是完全平方数. 对于任意的$z \in \mathbb{Q}[\sqrt{d}]$, 定义$N(z) = z \cdot \bar{z}$.

        先证明对任意的$a, b \in \mathbb{Q}[\sqrt{d}]$, 有$N(a \cdot b) = N(a) \cdot N(b)$. 设$a = x_1 + y_1 \sqrt{d}$, $b = x_2 + y_2 \sqrt{d}$, 其中$x_1, y_1, x_2, y_2 \in \mathbb{Q}$. 则有
        $$N(a) = (x_1 + y_1 \sqrt{d})(x_1 - y_1 \sqrt{d}) = x_1^2 - d y_1^2,$$
        $$N(b) = (x_2 + y_2 \sqrt{d})(x_2 - y_2 \sqrt{d}) = x_2^2 - d y_2^2.$$
        因此,
        \begin{align*}
        N(a \cdot b) &= N((x_1 + y_1 \sqrt{d})(x_2 + y_2 \sqrt{d})) \\
        &= N((x_1 x_2 + y_1 y_2 d) + (x_1 y_2 + x_2 y_1) \sqrt{d}) \\
        &= (x_1 x_2 + y_1 y_2 d)^2 - d (x_1 y_2 + x_2 y_1)^2 \\
        &= (x_1^2 - d y_1^2)(x_2^2 - d y_2^2) = N(a) \cdot N(b).
        \end{align*}

        再证明$N(\mathbb{Z}[\sqrt{d}]) \subset \mathbb{Z}$. 对于任意的$z = x + y \sqrt{d} \in \mathbb{Z}[\sqrt{d}]$, 其中$x, y \in \mathbb{Z}$. 则有
        $$N(z) = (x + y \sqrt{d})(x - y \sqrt{d}) = x^2 - d y^2 \in \mathbb{Z}.$$
        因此, $N(\mathbb{Z}[\sqrt{d}]) \subset \mathbb{Z}$.
        下面证明$\mathbb{Z}[\sqrt{d}]^{\times} = \{z \in \mathbb{Z}[\sqrt{d}] \mid N(z) = \pm 1\}$.
        设$z \in \mathbb{Z}[\sqrt{d}]^{\times}$. 则存在$w \in \mathbb{Z}[\sqrt{d}]$, 使得$z \cdot w = 1$. 因此,
        $$N(z) \cdot N(w) = N(z \cdot w) = N(1) = 1.$$
        因为$N(z), N(w) \in \mathbb{Z}$, 所以$N(z) = \pm 1$.
        反之, 设$z \in \mathbb{Z}[\sqrt{d}]$, 且$N(z) = \pm 1$. 则有
        $$N(z) = z \cdot \bar{z} = \pm 1.$$
        因此, 
        $$z \cdot (\pm \bar{z}) = 1.$$
        所以, $z \in \mathbb{Z}[\sqrt{d}]^{\times}$.
        综上, $\mathbb{Z}[\sqrt{d}]^{\times} = \{z \in \mathbb{Z}[\sqrt{d}] \mid N(z) = \pm 1\}$.
    \end{proof}

    \begin{problem}[B5)]
        对于 $d<0$ ,试计算 $\mathbb{Z}[\sqrt{d}]^{\times}$。
        当 $d>0$ 时, $\mathbb{Z}[\sqrt{d}]^{\times}$的结构要复杂的多。实际上,

        $$
        N(z=x+y \sqrt{d})= \pm 1 \quad \Leftrightarrow \quad x^2-d y^2= \pm 1 \text {. }
        $$

        上述方程通常被称作是 Pell 方程。研究 $\mathbb{Z}[\sqrt{d}]^{\times}$可以给出以上方程所有的整数解。
    \end{problem}

    \begin{proof}
        设$d \in \mathbb{Z}$不是完全平方数.

        如果$d < 0$, 则对于任意的$z = x + y \sqrt{d} \in \mathbb{Z}[\sqrt{d}]^{\times}$, 有
        $$N(z) = x^2 - d y^2 = \pm 1.$$
        因为$d < 0$, 所以$x^2 + |d| y^2 = \pm 1$. 因此, $y = 0$且$x = \pm 1$. 所以, $\mathbb{Z}[\sqrt{d}]^{\times} = \{\pm 1\}$.

        如果$d > 0$, 则对于任意的$z = x + y \sqrt{d} \in \mathbb{Z}[\sqrt{d}]^{\times}$, 有
        $$N(z) = x^2 - d y^2 = \pm 1.$$
        上述方程通常被称作是Pell方程. 研究$\mathbb{Z}[\sqrt{d}]^{\times}$可以给出以上方程所有的整数解.
    \end{proof}

    \begin{problem}[B6)]
        证明, $\mathbb{Z}[\sqrt{2}]^{\times} \cap(1,3)=\{1+\sqrt{2}\}$ 。
    \end{problem}

    \begin{proof}
        设$z = x + y \sqrt{2} \in \mathbb{Z}[\sqrt{2}]^{\times} \cap (1, 3)$, 其中$x, y \in \mathbb{Z}$且$1 < z < 3$. 则有
        $$N(z) = x^2 - 2 y^2 = \pm 1.$$
        因为$1 < z < 3$, 所以$x + y \sqrt{2} < 3$. 因此, $x < 3 - y \sqrt{2}$. 因为$x, y \in \mathbb{Z}$, 所以$y$只能取0或1.

        如果$y = 0$, 则$x^2 = \pm 1$, 与$x \in \mathbb{Z}$矛盾. 如果$y = 1$, 则$x^2 - 2 = \pm 1$. 因此, $x^2 = 3$或$x^2 = 1$. 因为$x \in \mathbb{Z}$, 所以$x = 1$. 因此, $z = 1 + \sqrt{2}$.

        综上, $\mathbb{Z}[\sqrt{2}]^{\times} \cap (1, 3) = \{1 + \sqrt{2}\}$.
    \end{proof}

    \begin{problem}[B7)]
        证明, $\mathbb{Z}[\sqrt{2}]^{\times}=\left\{ \pm(1+\sqrt{2})^k \mid k \in \mathbb{Z}\right\}$ 并给出群同构 $\mathbb{Z}[\sqrt{2}]^{\times} \simeq \mathbb{Z} \times \mathbb{Z} / 2 \mathbb{Z}^{\circ}$
    \end{problem}

    \begin{proof}
        设$z = x + y \sqrt{2} \in \mathbb{Z}[\sqrt{2}]^{\times}$, 其中$x, y \in \mathbb{Z}$. 则有
        $$N(z) = x^2 - 2 y^2 = \pm 1.$$
        因为$x, y \in \mathbb{Z}$, 所以存在整数$k$, 使得$z = \pm (1 + \sqrt{2})^k$.

        定义映射$\phi: \mathbb{Z}[\sqrt{2}]^{\times} \rightarrow \mathbb{Z} \times \mathbb{Z} / 2 \mathbb{Z}^{\circ}$, 对于任意的$z = \pm (1 + \sqrt{2})^k \in \mathbb{Z}[\sqrt{2}]^{\times}$, 令
        $$\phi(z) = (k, [0])$$
        如果$z = (1 + \sqrt{2})^k$, 否则令
        $$\phi(z) = (k, [1]).$$
        下面验证$\phi$为群同构.

        (群同态) 对于任意的$z_1 = \pm (1 + \sqrt{2})^{k_1}, z_2 = \pm (1 + \sqrt{2})^{k_2} \in \mathbb{Z}[\sqrt{2}]^{\times}$, 有
        $$\phi(z_1 z_2) = \phi(\pm (1 + \sqrt{2})^{k_1 + k_2}) = (k_1 + k_2, [0])$$
        如果$z_1 z_2 = (1 + \sqrt{2})^{k_1 + k_2}$, 否则
        $$\phi(z_1 z_2) = (k_1 + k_2, [1]).$$
        因此,
        $$\phi(z_1) + \phi(z_2) = (k_1, [0]) + (k_2, [0]) = (k_1 + k_2, [0])$$
        如果$z_1 = (1 + \sqrt{2})^{k_1}$且$z_2 = (1 + \sqrt{2})^{k_2}$, 否则
        $$\phi(z_1) + \phi(z_2) = (k_1 + k_2, [1]).$$
        因此, $\phi$为群同态.

        (双射性) 如果$\phi(z_1) = \phi(z_2)$, 则$z_1$和$z_2$的指数相同且符号相同. 因此, $z_1 = z_2$. 所以, $\phi$为单射.
        对于任意的$(k, [0]) \in \mathbb{Z} \times \mathbb{Z} / 2 \mathbb{Z}^{\circ}$, 令$z = (1 + \sqrt{2})^k$. 则有
        $$\phi(z) = (k, [0]).$$
        对于任意的$(k, [1]) \in \mathbb{Z} \times \mathbb{Z} / 2 \mathbb{Z}^{\circ}$, 令$z = - (1 + \sqrt{2})^k$. 则有
        $$\phi(z) = (k, [1]).$$
        所以, $\phi$为满射.
        综上, $\phi$为群同构.
    \end{proof}

    \begin{problem}[B8)]
        如何刻画 Pell 方程 $x^2-2 y^2=1$ 和 $x^2-2 y^2=-1$ 的所有整数解?
    \end{problem}

    \begin{proof}
        Pell方程$x^2 - 2 y^2 = 1$的所有整数解可以表示为$(x_k, y_k)$, 其中
        $$x_k + y_k \sqrt{2} = (1 + \sqrt{2})^k,$$
        $k \in \mathbb{Z}$.

        Pell方程$x^2 - 2 y^2 = -1$的所有整数解可以表示为$(x_k, y_k)$, 其中
        $$x_k + y_k \sqrt{2} = (1 + \sqrt{2})^{2k + 1},$$
        $k \in \mathbb{Z}$.
    \end{proof}

    \begin{problem}[B9)]
        证明, 有序列 $\left\{z_n\right\}_{n \geqslant 1} \subset \mathbb{Z}[\sqrt{d}]-\{0\}$ ,使得 $\lim _{n \rightarrow \infty} z_n=0$ 而 $\left\{N\left(z_n\right)\right\}_{n \geqslant 1}$ 是有界的。
        (提示:使用 Dirichlet 引理:对任意 $\alpha \in \mathbb{R}$ 和正整数 $M$ ,存在整数 $p$ 和正整数 $q \leqslant M$ ,使得 $|p-q \alpha|<\frac{1}{M}$ 。这个引理可以用抽屉原理直接证明或者请从文献中查阅证明)
    \end{problem}

    \begin{proof}
        设$d \in \mathbb{Z}$不是完全平方数. 下面证明存在序列$\{z_n\}_{n \geqslant 1} \subset \mathbb{Z}[\sqrt{d}] - \{0\}$, 使得$\lim_{n \rightarrow \infty} z_n = 0$且$\{N(z_n)\}_{n \geqslant 1}$是有界的.

        对任意的$n \geqslant 1$, 令$M = n$. 根据Dirichlet引理, 存在整数$p_n$和正整数$q_n \leqslant M$, 使得
        $$|p_n - q_n \sqrt{d}| < \frac{1}{M}.$$
        定义$z_n = p_n + q_n \sqrt{d}$. 则有
        $$|z_n| = |p_n + q_n \sqrt{d}| < |p_n - q_n \sqrt{d}| + 2 |q_n| \sqrt{d} < \frac{1}{M} + 2 M \sqrt{d}.$$
        因此, $\lim_{n \rightarrow \infty} z_n = 0$.

        再计算$N(z_n)$:
        $$N(z_n) = (p_n + q_n \sqrt{d})(p_n - q_n \sqrt{d}) = p_n^2 - d q_n^2.$$
        因为$|p_n - q_n \sqrt{d}| < \frac{1}{M}$, 所以$p_n^2 - d q_n^2$是有界的. 因此, $\{N(z_n)\}_{n \geqslant 1}$是有界的.
    \end{proof}

    \begin{problem}[B10)]
        证明,存在上述序列的子序列 $\left\{w_n\right\}_n \geqslant 1$ 以及整数 $k$ ,使得对任意的 $n, m \geqslant 1$ ,我们有 $N\left(w_n\right)=k$ 并且 $w_n \bar{w}_m \in k \mathbb{Z}[\sqrt{d}]$ 。(提示:考虑 $w_n$ 在 $\mathbb{Z}[\sqrt{d}] / k \mathbb{Z}[\sqrt{d}]$ 中的像)
    \end{problem}

    \begin{proof}
        设$d \in \mathbb{Z}$不是完全平方数. 根据题意, 存在序列$\{z_n\}_{n \geqslant 1} \subset \mathbb{Z}[\sqrt{d}] - \{0\}$, 使得$\lim_{n \rightarrow \infty} z_n = 0$且$\{N(z_n)\}_{n \geqslant 1}$是有界的.

        因为$\{N(z_n)\}_{n \geqslant 1}$是有界的, 所以存在整数$k$, 使得无穷多个$z_n$满足$N(z_n) = k$. 设这些$z_n$构成子序列$\{w_n\}_{n \geqslant 1}$.

        下面证明对任意的$n, m \geqslant 1$, 有$w_n \bar{w}_m \in k \mathbb{Z}[\sqrt{d}]$.

        因为$N(w_n) = k$, 所以
        $$w_n \bar{w}_n = k.$$
        因此,
        $$w_n \bar{w}_m = w_n \bar{w}_n \cdot \frac{\bar{w}_m}{\bar{w}_n} = k \cdot \frac{\bar{w}_m}{\bar{w}_n}.$$
        因为$\frac{\bar{w}_m}{\bar{w}_n} \in \mathbb{Z}[\sqrt{d}]$, 所以$w_n \bar{w}_m \in k \mathbb{Z}[\sqrt{d}]$.

        综上, 存在上述序列的子序列$\{w_n\}_{n \geqslant 1}$以及整数$k$, 使得对任意的$n, m \geqslant 1$, 我们有$N(w_n) = k$并且$w_n \bar{w}_m \in k \mathbb{Z}[\sqrt{d}]$.
    \end{proof}

    \begin{problem}[B11)]
        证明, $\mathbb{Z}[\sqrt{d}]^{\times}$是无限集。
    \end{problem}

    \begin{proof}
        设$d \in \mathbb{Z}$不是完全平方数. 下面证明$\mathbb{Z}[\sqrt{d}]^{\times}$是无限集.

        根据题意, 存在序列$\{w_n\}_{n \geqslant 1} \subset \mathbb{Z}[\sqrt{d}] - \{0\}$以及整数$k$, 使得对任意的$n, m \geqslant 1$, 我们有$N(w_n) = k$并且$w_n \bar{w}_m \in k \mathbb{Z}[\sqrt{d}]$.

        因为对任意的$n, m \geqslant 1$, 有$w_n \bar{w}_m \in k \mathbb{Z}[\sqrt{d}]$, 所以存在整数$l_{n, m}$, 使得
        $$w_n \bar{w}_m = k l_{n, m}.$$
        因此,
        $$\frac{w_n}{k} = l_{n, m} \cdot \frac{\bar{w}_m}{k}.$$
        因为$\frac{\bar{w}_m}{k} \in \mathbb{Z}[\sqrt{d}]$, 所以$\frac{w_n}{k} \in \mathbb{Z}[\sqrt{d}]$.

        因为对任意的$n \geqslant 1$, 有$N(w_n) = k$, 所以
        $$N\left(\frac{w_n}{k}\right) = \frac{N(w_n)}{k^2} = \frac{k}{k^2} = \frac{1}{k}.$$
        因此, $\frac{w_n}{k} \in \mathbb{Z}[\sqrt{d}]^{\times}$.

        由于序列$\{w_n\}_{n \geqslant 1}$是无限的, 所以$\mathbb{Z}[\sqrt{d}]^{\times}$也是无限的.
    \end{proof}

    \begin{problem}[B12)]
        证明, $\mathbb{Z}[\sqrt{d}]^{\times}$是无限集。
    \end{problem}

    \begin{proof}
        设$d \in \mathbb{Z}$不是完全平方数. 下面证明$\mathbb{Z}[\sqrt{d}]^{\times}$是无限集.

        根据题意, 存在序列$\{w_n\}_{n \geqslant 1} \subset \mathbb{Z}[\sqrt{d}] - \{0\}$以及整数$k$, 使得对任意的$n, m \geqslant 1$, 我们有$N(w_n) = k$并且$w_n \bar{w}_m \in k \mathbb{Z}[\sqrt{d}]$.

        因为对任意的$n, m \geqslant 1$, 有$w_n \bar{w}_m \in k \mathbb{Z}[\sqrt{d}]$, 所以存在整数$l_{n, m}$, 使得
        $$w_n \bar{w}_m = k l_{n, m}.$$
        因此,
        $$\frac{w_n}{k} = l_{n, m} \cdot \frac{\bar{w}_m}{k}.$$
        因为$\frac{\bar{w}_m}{k} \in \mathbb{Z}[\sqrt{d}]$, 所以$\frac{w_n}{k} \in \mathbb{Z}[\sqrt{d}]$.

        因为对任意的$n \geqslant 1$, 有$N(w_n) = k$, 所以
        $$N\left(\frac{w_n}{k}\right) = \frac{N(w_n)}{k^2} = \frac{k}{k^2} = \frac{1}{k}.$$
        因此, $\frac{w_n}{k} \in \mathbb{Z}[\sqrt{d}]^{\times}$.

        由于序列$\{w_n\}_{n \geqslant 1}$是无限的, 所以$\mathbb{Z}[\sqrt{d}]^{\times}$也是无限的.
    \end{proof}

    \begin{problem}[B13)]
        证明,存在 $\eta_d \in(1, \infty)$(被称作是基本单位),使得 $\eta_d$ 生成了 $\mathbb{Z}[\sqrt{d}]^{\times} \cap(0, \infty)$ 。特别地, $\mathbb{Z}[\sqrt{d}]^{\times} \simeq \mathbb{Z} \times \mathbb{Z} / 2 \mathbb{Z}^{\circ}$
        (注意到:对任意的 $u \in \mathbb{Z}[\sqrt{d}]^{\times}-\{ \pm 1\}$ ,四个点 $\pm u, \pm \bar{u}$ 在区间 $(-\infty,-1),(-1,0),(0,1),(1, \infty)$ 中各有一个)
    \end{problem}

    \begin{proof}
        设$d \in \mathbb{Z}$不是完全平方数. 下面证明存在$\eta_d \in (1, \infty)$(被称作是基本单位), 使得$\eta_d$生成了$\mathbb{Z}[\sqrt{d}]^{\times} \cap (0, \infty)$.

        因为$\mathbb{Z}[\sqrt{d}]^{\times}$是无限集, 所以存在$u \in \mathbb{Z}[\sqrt{d}]^{\times}$, 使得$u > 1$. 定义
        $$\eta_d = \min\{u \in \mathbb{Z}[\sqrt{d}]^{\times} \mid u > 1\}.$$
        则有$\eta_d \in (1, \infty)$.

        下面证明$\eta_d$生成了$\mathbb{Z}[\sqrt{d}]^{\times} \cap (0, \infty)$.

        设$v \in \mathbb{Z}[\sqrt{d}]^{\times} \cap (0, \infty)$. 则存在整数$k$, 使得
        $$\eta_d^k \leqslant v < \eta_d^{k + 1}.$$
        因此,
        $$1 \leqslant \frac{v}{\eta_d^k} < \eta_d.$$
        因为$\eta_d$为最小的单位, 所以$\frac{v}{\eta_d^k} = 1$. 因此, $v = \eta_d^k$.

        综上, 存在$\eta_d \in (1, \infty)$(被称作是基本单位), 使得$\eta_d$生成了$\mathbb{Z}[\sqrt{d}]^{\times} \cap (0, \infty)$. 特别地, 有群同构$\mathbb{Z}[\sqrt{d}]^{\times} \simeq \mathbb{Z} \times \mathbb{Z} / 2 \mathbb{Z}^{\circ}$.
    \end{proof}

\newpage


\end{document}
