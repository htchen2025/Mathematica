\documentclass[a4paper, 12pt, UTF8, heading=true, scheme=chinese]{ctexart}

\usepackage[a4paper, left=2cm, right=2cm, top=2cm, bottom=2cm]{geometry}
\usepackage{amsmath,amssymb,bm,graphicx,xcolor,tikz,array,booktabs,multicol,multirow,titlesec,hyperref,biblatex,algorithm2e,listings,mathtools}

\titleformat{\section}
  {\normalfont\Large\bfseries\raggedright}
  {}
  {0pt}
  {}

\newcommand{\R}{\mathbb{R}}
\newcommand{\Z}{\mathbb{Z}}
\newcommand{\N}{\mathbb{N}}
\newcommand{\Q}{\mathbb{Q}}
\newcommand{\C}{\mathbb{C}}
\newcommand{\st}{\text{s.t.}}

\newenvironment{solution}{\par\noindent\textbf{解:}\par}{\hfill$\square$\par}
\newenvironment{proof}{\par\noindent\textbf{证明:}}{\hfill$\square$\par}
\newenvironment{problem}[1][]{\par\noindent\textmd{#1}\par}{}
\newenvironment{note}{\par\noindent\textbf{注:}}{\par}
\newenvironment{tip}{\par\noindent\textbf{提示:}}{\par}

\linespread{1.5}

\title{\textbf{群 与 Galois 理论 \\ 作业3}}
\author{陈宏泰 \\ 2024011131 \\ 清华大学数学科学系 \\ \texttt{cht24@mails.tsinghua.edu.cn}}
\date{\today}

\begin{document}

\maketitle 

\tableofcontents
\newpage

\section{A. \texorpdfstring{$60$}.阶的单群}

    \begin{problem}[A1)]
    \end{problem}

    \begin{proof}
        由Sylow第二定理, 
        $$
        s_5\equiv 1\pmod 5,
        $$
        故$s_5$可能的取值为1,6,11,16,21,26,31,36,41,46,51,56.
        同时$s_5\big| |G|=60$, 故$s_5=1$或6. 由假设知$s_5\neq 1$, 所以$s_5=6$.
    \end{proof}

    \begin{problem}[A2)]
    \end{problem}

    \begin{proof}
        (a)

        由于课上命题, 通过对大群的Sylow $p$-子群共轭可以得到子群的Sylow $p$-子群.
        于是$\exists\ g\in G,\ \st H\cap gS_1g^{-1}$是$H$的Sylow $p$-子群.
        由于$|H|$是5的倍数, $H$的Sylow $p$-子群的阶为5, 所以$gS_1g^{-1}\subset H$.
        又由Sylow第二定理, $\exists\ g_i\in G,\ \st S_i=g_iS_1g_i^{-1}$, 其中$i\in \{2,3,4,5,6\}$.
        由于$H$是正规子群, 于是$S_i\subset H,\forall\ i$.
        可得$|H|\geqslant 1+(5-1)\times6=25$. 又$|H|\big|60$, 得到$|H|=30$.

        (b)

        由(a)中证明过程, $H$有1个单位元, $(5-1)\times6=24$个阶为5的元素有$(5-1)\times6=24$个, 
        于是其他阶的元素只剩下5个, 有$(3-1)s_3\leq 5$.
        由Sylow第二定理, 
        $$
        s_3\equiv 1\pmod 3,
        $$
        于是$s_3$可能的取值为1,4,7,10,$\cdots$. 于是$s_1=1$. $H$只有一个Sylow 3-子群$T$.

        (c)

        $\forall\ s_1t_1, s_2t_2\in S_1T$, 如果$s_1t_1=s_2t_2$, 那么$s_2^{-1}s_1=t_2t_1^{-1}$.
        因为$s_2^{-1}s_1\in S_1$, $t_2t_1^{-1}\in T$, $S_1\cap T=\{1_{_H}\}$, 
        所以$s_2^{-1}s_1=t_2t_1^{-1}=1_{_H}\Rightarrow s_1=s_2, t_1=t_2$.
        故$\forall\ s_1\neq s_2\in S_1, t_1\neq t_2\in T, s_1t_1\neq s_2t_2$.
        于是$|S_1T|=15$. 

        由于$H$只有一个Sylow 3-子群$T$, 故$T\triangleleft H$.
        $\forall\ s_1t_1, s_2t_2\in S_1T$, 有$(t_1t_2^{-1})s_2=s_3t_3\in S_1T$, 于是
        $$
        (s_1t_1)(s_2t_2)^{-1}=s_1(t_1t_2^{-1}s_2)=s_1(s_3t_3)=(s_1s_3)t_3\in S_1T.
        $$
        故$S_1T$是$H$的子群, 且阶为15.

        (d)

        $S_1T$在$H$中的指数为2, 故$S_1T$是$H$的正规子群. 由于$S_1\subset S_1T$, 故$S_i=g_iS_1g_i^{-1}\subset S_1T$.
        但同时对于$S_1T$, 有$(5-1)s_5\leq 15$, 故$S_1T$只有1个Sylow 5-子群. 
        即所有$S_i,\forall\ i\in\{1,2,3,4,5,6\}$都是相同的, 故$H$将只有1个Sylow 5-子群.
        而这与(a)中结论矛盾.

        综上, $|H|$不是5的倍数.


    \end{proof}

    \begin{problem}[A3)]
    \end{problem}

    \begin{proof}
        (a)

        $G/H$的阶只能是15,20,30. 对于$G/H$, $(5-1)s_5\leqslant |G/H|-1$.

        如果$|G/H|=15$, $s_5<4$, 故$s_5=1$.

        如果$|G/H|=20$, $s_5<5$, 故$s_5=1$.

        如果$|G/H|=30$, $s_5<8$, 故$s_5=1$或6.
        如果$s_5=6$, 类似于A2)(b)(c)(d)的讨论, 可以推出矛盾.故$s_5=1$.

        综上, $s_5=1$, $G/H$只有一个Sylow 5-子群.

        (b)

        取$G/H$的Sylow 5-子群$S=\{H=g_0H,g_1H,g_2H,g_3H,g_4H\}$, 那么令
        $$
        H'=\bigcup_{i=0}^4 g_iH, 
        $$
        我们断言这是$G$的一个正规子群.下面证明:
        
        首先, $\forall\ h_1'=g_ih_1, h_2'=g_jh_2\in H'$, 
        $$
        h_1'(h_2')^{-1}=(g_ih_1)(g_jh_2)^{-1}=g_i(h_1h_2^{-1})g_j=g_i(g_jh_3)\in (g_ig_j)H\subset H',
        $$
        于是$H'$是$G$的子群. 而$\forall\ g\in G$, 有$gSg^{-1}=S\Rightarrow \forall\ h'\in H', gh'g^{-1}\in H'$.
        于是$H'\triangleleft G$, 且$H'\neq G$, $|H'|$是5的倍数. 由A2)推出矛盾.
    \end{proof}

    \begin{problem}[A4)]
    \end{problem}

    \begin{proof}
        对于$|H|=6$的情况, $s_2=1,3,\cdots, s_3=1,4,\cdots,s_2|6,s_3|6$,  
        并且$s_2+2s_3\leqslant5$, 只能有$s_3=1,s_2=1$或3.

        对于$|H|=12$的情况, $s_2=1,3,\cdots, s_3=1,4,\cdots,s_2|12,s_3|12$, 
        并且$3s_2+2s_3\leqslant11$, 只能有$s_2=1, s_3=1$; $s_2=3, s_3=1$或者$s_2=1, s_3=4$.

        综上, $H$只有一个Sylow 2-子群或只有一个Sylow 3-子群. 不妨记为$H'$其为Sylow $p$-子群, 有$H'\triangleleft H$, $|H'|\leqslant4$.
        $\forall\ g\in G,h'\in H'$, 有$gh'g^{-1}\in H$. 并且由于$gh'g^{-1}$与$h'$在$H$中的阶相同, 于是$|gh'g^{-1}|\big|p^2$.
        由于$H'$是$H$中唯一的Sylow $p$-子群, 于是$gh'g^{-1}\in H'$. 这推出$H'\triangleleft G$, 由A3)可知矛盾呢.
    \end{proof}

    \begin{problem}[A5)]
    \end{problem}

    \begin{proof}
        考虑$G$在$G/H$的作用, 这显然是传递的. 这又诱导了一个群同态$\varphi:$
        $$
        G\to \mathfrak{S}_{G/H}\cong \mathfrak{S}_d,
        $$
        其中$d=|G/H|=[G:H]$.
        由于$\ker\varphi\triangleleft G$以及$G$是单群, $\ker\varphi={1}$(显然$\ker\varphi\neq G$). 于是$\varphi$是一个单射.
        从而有$|G|\big|d!$, 于是$d\geqslant5$. 

        如果$d=5$, 可以将$G$视为$\mathfrak{S}_5$的一个子群, 此时$[\mathfrak{S}_5:G]=2$, 于是$G\triangleleft \mathfrak{S}_5$. 
        由作业二知$G\simeq \mathfrak{A}_5$.
    \end{proof}

    \begin{problem}[A6)]
    \end{problem}

    \begin{proof}
        考虑$G$在Sylow p-子群的集合$X$上的作用:
        $$
        G\times X\to X,\quad (g,S_i)\mapsto gS_ig^{-1},
        $$
        $X$记为$\{S_1,S_2,\cdots,S_{s_p}\}$. 于是由Sylow $p$-子群的共轭性质, 其诱导了一个群同态$\varphi:$
        $$
        G\to \mathfrak{S}_{_X}\cong \mathfrak{S}_{s_p},\quad g\mapsto\{S_i\mapsto gS_ig^{-1}\}.
        $$
        同A5)的证明, 可以得到$\ker\varphi=\{1\}$, 且$|X|\geqslant5$, 即$s_p\geqslant5$.
        再结合$s_p\equiv 1(\pmod p),s_p|60$, 可以得到$s_2\in\{5,15\},s_3=10,s_5=6$.
    \end{proof}

    \begin{problem}[A7)]
    \end{problem}

    \begin{proof}
        如果$s_2=5$, 那么由A6)中的$\varphi$知$G<\mathfrak{S}_5\Rightarrow G\simeq \mathfrak{A}_5$.
    \end{proof}

    \begin{problem}[A8)]
    \end{problem}

    \begin{proof}
        首先我们有, 两个$p$阶循环群, 除去单位元之外无共同元素. 
        假设$R$, $S$是两个阶为$p$的循环群, 且有非单位元的的共同元素$r$, 
        那么$R=\{1,r,r^2,\cdots,r^{p-1}\}=S$.
        
        假设每两个Sylow 2-子群共同元素只有单位元, 那么$G$中将有$1+3s_2+2s_3+4s_5=90$个不同的元素, 这与$|G|=60$矛盾.
        于是存在两个不同的Sylow 2-子群$P,Q$有共同的非单位元的元素. 
        并且每个Sylow 2-子群都共轭, 所以结构相同, 又不是循环群, 故都是Klein四元群. 
        Klein四元数群中每个元素都为2阶, 且这是交换群.
        
        接下来考虑$|P\cap Q|$. $|P\cap Q|=4$导致$P=Q$. 
        若$|P\cap Q|=3$, 由Klein四元数群的结构, 剩下一个不同元素又可以由两个共同的非单位元元素生成, 导致$|P\cap Q|=4$, 矛盾.
        于是$|P\cap Q|=2$.

        现在记$D=P\cap Q=\{1,s\}$, 这是$G$的一个2阶子群. 
        注意到$P$是Klein四元群从而是交换的, 故而$P$中元素都与$s$交换, 从而$R<N_G(D)$. 
        于是$4\big||N_G(D)|$, 又$|N_G(D)|\big|60$, $P\cup Q\subset N_G(D)\Rightarrow |N_G(D)|\geqslant6$, 故$|N_G(D)|=12$或20.

        又$N_G(D)<G$, 由A5), $[G:N_G(D)]\geqslant5\Rightarrow |N_G(D)|\leqslant12$. 
        
        故只能有$|N_G(D)|=12, [G:N_G(D)]=5$.
        意味着$G\simeq \mathfrak{A}_5$.
    \end{proof}

    \begin{problem}[A9)]
    \end{problem}

    \begin{proof}
        由A7), A8), 无论$s_2$取值为5还是15, 都有$G\simeq \mathfrak{A}_5$.

        只需计算$\mathfrak{A}_5$的$s_2$即可.
        由A8)讨论知, 其中Sylow 2-子群都为Klein四元群. 只需考察(12)(34)在哪些Klein四元群中.
        有以下几种不同的情况:
        \begin{align*}
            &((13)(24))((12)(34))=(14)(23),\\
            &((12)(35))((12)(34))=(345),\\
            &((13)(25))((12)(34))=(15234).
        \end{align*}
        只有第一种情况, (12)(34), (13)(24), (14)(23)构成Klein四元群. 故Sylow 2-子群共有15/3=5个, 即$s_2=5$.
    \end{proof}
\newpage

\section{B. 与Sylow \texorpdfstring{$p$}.-子群相关的补充}

    \begin{problem}[B1)]
    \end{problem}

    \begin{proof}
        考虑$S$在$G/N_G(S)$上的作用:
        $$
        S\times G/N_G(S)\to G/N_G(S),\quad (g,hN_G(S))\mapsto ghg^{-1}N_G(S).  
        $$
        注意到$N_G(S)\in G/N_G(S)$被$S$固定, 而对于其他$hN_G(S)\in G/N_G(S)$, 有$ghg^{-1}\notin N_G(S)$, 
        否则$h\in g^{-1}N_G(S)g\Rightarrow h\in N_G(S)\Rightarrow hN_G(S)=N_G(S)\in G/N_G(S)$.
        于是由讲义引理 28, $[G:N_G(S)]\equiv1\pmod p$.
    \end{proof}

    \begin{problem}[B2)]
    \end{problem}

    \begin{proof}
        由Sylow定理可知, $\exists\ g\in G,\st\ H\cap gSg^{-1}$是$H$的Sylow $p$-子群.
        于是自然地有$g^{-1}(H\cap gSg^{-1})g=g^{-1}Hg\cap S$是$g^{-1}Hg$的Sylow $p$-子群.
        又$H\triangleleft G$, 于是$g^{-1}Hg=H$. 故$H\cap S$是$H$的Sylow $p$-子群.
    \end{proof}

    \begin{problem}[B3)]
    \end{problem}

    \begin{proof}
        设$|G|=p^am$, 其中$p\nmid m$, 则$|S|=p^a$.
        设$|H|=p^cn$, 其中$p\nmid n$.
        由于$H\triangleleft G$, 于是$|G/H|=p^{a-c}(m/n)$, 由于$p\nmid (m/n)$, 故$G/H$的Sylow $p$-子群的阶为$p^{a-c}$.

        一方面:
        考虑$S\cap H$, 由B2)结论知, $|S\cap H|=p^c$. 由于$H\triangleleft G$, 有$S\cap H\triangleleft S$. 
        考虑$\pi$在$S$上的限制$\pi_1:S\to G/H$, 有$\ker\pi_1=\{s\in S\mid sH=H\}=\{s\in S\mid s\in H\}=S\cap H$.
        由第一同构定理, $S/S\cap H\simeq \pi_1(S)=\pi(S)$. 
        于是$|\pi(S)|=|S/S\cap H|=|S|/|S\cap H|=p^a/p^c=p^{a-c}$, 这表明$\pi(S)$是$G/H$的Sylow $p$-子群.

        反之: 如果$S'<G/H$是Sylow $p$-子群, 考虑$S'$的原像$K=\pi^{-1}(S')$, 
        则$K$显然是$G$的一个子群, 并且$|K|=|S'||H|=p^{a-c}p^cn=p^an$.
        由Sylow第一定理, $K$有Sylow $p$-子群$S$, 于是这也是$G$的Sylow $p$-子群.
        显然有$\pi(S)<S'$, 另一方面由前可知$\pi(S)$也是$G/H$的Sylow $p$-子群, 故而$|\pi(S)|=p^{a-c}=S'$, 从而$\pi(S)=S'$.
    \end{proof}

    \begin{problem}[B4)]
    \end{problem}

    \begin{proof}
        \begin{itemize}
            \item[$\bullet$]
                % $\prescript{G\curvearrowright}{}X$
                显然有$H\cdot \text{Stab}_G(x)\subset G$. $\forall\ g\in G$, 由于$\prescript{H\curvearrowright}{}X$是传递的, 
                $\exists\ h\in H,\st\ g(x)=h(x)$, 于是$h^{-1}g(x)=x\Rightarrow h^{-1}g\in \text{Stab}_G(x)$.
                从而$g\in h\text{Stab}_G(x)\Rightarrow G\subset H\cdot \text{Stab}_G(x)$. 综上, $G=H\text{Stab}_G(x)$.
            \item[$\bullet$] 
                考虑$H$所有Sylow p-子群构成的集合$X=\{S_i\mid S_i<H\text{为Sylow $p$-子群}\}$, 考虑$G$在$X$上的共轭作用:
                $$
                G\times X\to X,\quad (g, S_i)\mapsto gS_ig^{-1}.
                $$
                先说明这个作用是良定义的. 由于$S_i\subset H$, 又$H\triangleleft G$, 所以$gS_ig^{-1}\subset H$.
                又由Sylow定理, $gS_ig^{-1}\in X$, 从而这个共轭作用是良定义的. 
                
                同时Sylow定理第二定理指出$\prescript{H\curvearrowright}{}X$是传递的.
                对于这个作用, $\text{Stab}_G(S)=N_G(S)$, 故由前一问知$G=H\cdot N_G(S)$.
        \end{itemize}
    \end{proof}

    \begin{problem}[B5)]
    \end{problem}

    \begin{proof}
        由于$S<G$是一个Sylow $p$-子群, 又$S<N_G(S)\subset H<G$, 于是$S$也是$H$的Sylow $p$-子群.
        $\forall\ g\in N_G(H), gSg^{-1}\subset gHg^{-1}=H$. 故$gSg^{-1}$也是$H$的一个Sylow $p$-子群.
        由Sylow第二定理, $\exists\ h\in H,\st\ hSh^{-1}=gSg^{-1}$, 即$h^{-1}gSg^{-1}h=S\Rightarrow h^{-1}g\in N_G(S)$.
        从而$g\in hN_G(S)\subset hH=H$, 即$N_G(H)\subset H$. 结合$H\subset N_G(H)$, 有$H=N_G(H)$.
    \end{proof}

    \begin{problem}[B6)]
    \end{problem}

    \begin{proof}
        在B5)中令$H=N_G(S)$即得$N_G(H)=N_G(N_G(H))$.
    \end{proof}
\end{document}
