\documentclass[a4paper, 12pt, UTF8, heading=true, scheme=chinese]{ctexart}

\usepackage[a4paper, left=2cm, right=2cm, top=2cm, bottom=2cm]{geometry}
\usepackage{amsmath,amssymb,bm,graphicx,xcolor,tikz,array,booktabs,multicol,multirow,titlesec,hyperref,biblatex,algorithm2e,listings,mathtools}

\titleformat{\section}
  {\normalfont\Large\bfseries\raggedright}
  {}
  {0pt}
  {}

\newcommand{\R}{\mathbb{R}}
\newcommand{\Z}{\mathbb{Z}}
\newcommand{\N}{\mathbb{N}}
\newcommand{\Q}{\mathbb{Q}}
\newcommand{\C}{\mathbb{C}}
\newcommand{\st}{\text{s.t.}}

\newenvironment{solution}{\par\noindent\textbf{解:}\par}{\hfill$\square$\par}
\newenvironment{proof}{\par\noindent\textbf{证明:}}{\hfill$\square$\par}
\newenvironment{problem}[1][]{\par\noindent\textmd{#1}\kaishu}{\par}
\newenvironment{note}{\par\noindent\textbf{注:}}{\par}
\newenvironment{tip}{\par\noindent\textbf{提示:}}{\par}

\linespread{1.5}

\title{\textbf{ODE \\ 第九次作业}}
\author{陈宏泰 \\ 2024011131 \\ 清华大学数学科学系 \\ \texttt{cht24@mails.tsinghua.edu.cn}}
\date{\today}

\begin{document}

\maketitle 

\begin{problem}[\heiti 习题 $\mathbf{1}$]
	考虑两个一维向量场 $F:(0,1) \rightarrow \mathbb{R} ; G:(0, \infty) \rightarrow \mathbb{R}$
	$$
	F(r)=\frac{r\left(1-r^2\right)}{2} ; \quad G(r)=\frac{r}{2} .
	$$

	1) 证明 $\dot{r}=F(r)$ 与 $\dot{r}=G(r)$ 的相流均存在,并分别计算它们的相流.

	2) 证明 $\dot{r}=F(r)$ 与 $\dot{r}=G(r)$ 拓扑共轭.
	
	3) 证明 $\dot{X}=F(X)$ 与 $\dot{Y}=G(Y)$ 在平衡点附近局部拓扑共轭,这里
	$$
	\begin{aligned}
	& F(x, y):=\left(x / 2-y-x\left(x^2+y^2\right) / 2, x+y / 2-y\left(x^2+y^2\right) / 2\right) ; \\
	& G(x, y):=(x / 2-y, x+y / 2) .
	\end{aligned}
	$$
\end{problem}

\begin{solution}
	1) 对于 $\dot{r}=F(r)$,分离变量得
	$$
	\int \frac{1}{r(1-r^2)} d r=\int \frac{1}{r}+\frac{1}{2(1-r)}-\frac{1}{2(1+r)} d r=\ln |r|-\frac{1}{2} \ln |1-r^2|=t+C .
	$$
	因此相流为
	$$
	\varphi_{t}(r)=\sqrt{\frac{r^{2} e^{t}}{1+r^{2}\left(e^{t}-1\right)}} .
	$$

	对于 $\dot{r}=G(r)$,同样分离变量得
	$$
	\int \frac{1}{r} d r=\int \frac{1}{2} d t \Rightarrow \ln |r|=\frac{t}{2}+C .
	$$
	因此相流为
	$$
	\psi_{t}(r)=r e^{t / 2} .
	$$

	2) 取同胚映射 $h:(0,1) \rightarrow(0, \infty), h(r)=\frac{r}{\sqrt{1-r^{2}}}$,则有
	$$
	h\left(\varphi_{t}(r)\right)=\frac{\sqrt{\frac{r^{2} e^{t}}{1+r^{2}\left(e^{t}-1\right)}}}{\sqrt{1-\frac{r^{2} e^{t}}{1+r^{2}\left(e^{t}-1\right)}}}=\frac{r e^{t / 2}}{\sqrt{1-r^{2}}}=\psi_{t}\left(\frac{r}{\sqrt{1-r^{2}}}\right)=\psi_{t}(h(r)) .
	$$

	3) 计算 $F$ 在原点的雅可比矩阵为
	$$
	D F(0,0)=\left(\begin{array}{cc}
	1 / 2 & -1 \\
	1 & 1 / 2
	\end{array}\right) ,
	$$
	其特征值为 $\lambda_{1}=3 / 2 i, \lambda_{2}=-3 / 2 i$,为纯虚特征值.计算 $G$ 在原点的雅可比矩阵为
	$$
	D G(0,0)=\left(\begin{array}{cc}
	1 / 2 & -1 \\
	1 & 1 / 2
	\end{array}\right) ,
	$$
	其特征值同样为 $\lambda_{1}=3 / 2 i, \lambda_{2}=-3 / 2 i$.由课本定理 6.5.1 可知 $\dot{X}=F(X)$ 与 $\dot{Y}=G(Y)$ 在平衡点附近局部拓扑共轭.	
\end{solution}

\begin{problem}[\heiti 习题 $\mathbf{2}$]
	设 $F \in C^1\left(\mathbb{R}^2, \mathbb{R}^2\right)$ ,且 $F(0)=0, F^{\prime}(0)=D$ ,其中
	$$
	D=\left[\begin{array}{ll}
	\lambda & \\
	& \mu
	\end{array}\right] ; \quad \lambda \leq \mu<0 .
	$$
	本题的目的是证明 $\dot{X}=F(X)$ 与 $\dot{Y}=D Y$ 在平衡点附近局部拓扑共轭.我们用 $\phi(t, z)$ 表示初值问题
	$$
	\dot{X}=F(X) ; \quad X(0)=z
	$$
	的极大解,其定义域记为 $I_z$ .承认(以后会证明)$\phi$ 作为 $(t, z)$ 的映射在其定义域上为 $C^1$ 光滑.

	1) 证明存在 $\rho>0$ 使得对任意的 $r \in(0, \rho]$ ,向量场 $F(X)$ 限制在 $$S_r:= \left\{X \in \mathbb{R}^2:|X|=r\right\}$$ 上朝向圆内,即
	$$
	F(X) \cdot X<0, \quad \forall X \in S_r .
	$$

	2) 固定 $r \in(0, \rho)$ .对任意的 $z \in B_\rho:=\{X:|X|<\rho\}$ 且 $z \neq 0$ ,证明存在 $T(z) \in I_z$ 使得 $\phi(T(z), z) \in S_r$ .

	3) 证明 $T: B_\rho \backslash\{0\} \rightarrow \mathbb{R}$ 连续.

	4) 定义 $h: B_\rho \rightarrow \mathbb{R}^2$ 为 $h(0)=0$ ,
	$$
	h(z):=e^{-T(z) D} \phi(T(z), z), \quad z \in B_\rho \backslash\{0\} .
	$$
	证明 $h$ 连续.

	5) 证明 $h\left(B_\rho\right)$ 为开集,且 $h: B_\rho \rightarrow h\left(B_\rho\right)$ 为同胚.

	6) 证明 $h$ 为 $\dot{X}=F(X)$ 与 $\dot{Y}=D Y$ 的局部拓扑共轭,即对任意的 $z \in B_\rho$ 以及 $t \in I_z$ 使得 $\phi(t, z) \in B_\rho$ 有
	$$
	h(\phi(t, z))=e^{t D} h(z) .
	$$
\end{problem}

\begin{solution}
	1) 由于 $F$ 在原点处可微,故存在 $\delta>0$ 使得当 $|X|<\delta$ 时有
	$$
	F(X)=D X+|X| \varepsilon(X) ,
	$$
	其中 $\lim _{X \rightarrow 0} \varepsilon(X)=0$.取 $\rho=\min \left\{\delta, \frac{|\lambda|}{2}\right\}$ ,则当 $|X|=r \leq \rho$ 时,
	$$
	F(X) \cdot X=X^T D X+|X| \varepsilon(X) \cdot X \leq \lambda|X|^2+|X|^2|\varepsilon(X)|<0 .
	$$

	2) 对任意的 $z \in B_\rho \backslash\{0\}$ ,由第 1 小题可知 $\frac{d}{d t}|\phi(t, z)|^2=2 F(\phi(t, z)) \cdot \phi(t, z)<0$ ,因此存在唯一的 $T(z)>0$ 使得 $|\phi(T(z), z)|=r$ .

	3) 设 $z_n \rightarrow z$ ,则由 $\phi$ 的连续性可知 $\phi(T(z_n), z_n) \rightarrow \phi(T(z), z)$ .又由于 $|\phi(T(z_n), z_n)|=r$ ,故 $|\phi(T(z), z)|=r$ ,由第 2 小题的唯一性可知 $T(z_n) \rightarrow T(z)$ .

	4) 由第 3 小题可知 $h$ 在 $B_\rho \backslash\{0\}$ 上连续.下面证明 $h$ 在原点处连续.对任意的 $\varepsilon>0$ ,取 $\delta=\min \left\{\rho, \frac{\varepsilon}{2}\right\}$ .当 $|z|<\delta$ 时,
	$$
	|h(z)|=\left|e^{-T(z) D} \phi(T(z), z)\right| \leq e^{|\mu| T(z)}|\phi(T(z), z)|=e^{|\mu| T(z)} r .
	$$
	又由于 $\lim _{z \rightarrow 0} T(z)=0$ ,故当 $|z|<\delta$ 足够小时有 $e^{|\mu| T(z)}<2$ ,因此 $|h(z)|<\varepsilon$ .

	5) 取任意的 $y_0 \in h\left(B_\rho\right)$ ,则存在 $z_0 \in B_\rho$ 使得 $h(z_0)=y_0$ .取 $\varepsilon>0$ 使得 $B_\varepsilon\left(z_0\right) \subset B_\rho$ .由 $h$ 在 $B_\rho$ 上连续可知存在 $\delta>0$ 使得当 $|z-z_0|<\delta$ 时有 $|h(z)-h\left(z_0\right)|<\varepsilon$ ,即 $h\left(B_\delta\left(z_0\right)\right) \subset B_\varepsilon\left(y_0\right)$ .因此 $h\left(B_\rho\right)$ 为开集.下面证明 $h: B_\rho \rightarrow h\left(B_\rho\right)$ 为同胚.取任意的 $y \in h\left(B_\rho\right)$ ,则存在唯一的 $z \in B_\rho$ 使得 $h(z)=y$ .因此只需证明 $h$ 为单射即可.设 $h(z_1)=h(z_2)$ ,则
	$$
	e^{-T\left(z_1\right) D} \phi\left(T\left(z_1\right), z_1\right)=e^{-T\left(z_2\right) D} \phi\left(T\left(z_2\right), z_2\right) .
	$$
	不失一般性,设 $T\left(z_1\right) \leq T\left(z_2\right)$ ,则有
	$$
	\phi\left(T\left(z_2\right)-T\left(z_1\right), \phi\left(T\left(z_1\right), z_1\right)\right)=\phi\left(T\left(z_2\right), z_2\right) .
	$$
	由 $\phi$ 的唯一性可知 $z_1=z_2$ .
	
	6) 对任意的 $z \in B_\rho$ 以及 $t \in I_z$ 使得 $\phi(t, z) \in B_\rho$ ,由定义有
	$$
	h(\phi(t, z))=e^{-T(\phi(t, z)) D} \phi(T(\phi(t, z)), \phi(t, z)) .
	$$
	又由于
	$$
	\phi(T(\phi(t, z)), \phi(t, z))=\phi(T(\phi(t, z))+t, z) ,
	$$
	且由定义可知 $T(\phi(t, z))+t=T(z)$ ,因此
	$$
	h(\phi(t, z))=e^{-T(z) D} \phi(T(z)+t, z)=e^{t D} e^{-T(z) D} \phi(T(z), z)=e^{t D} h(z) .
	$$
\end{solution}

\end{document}
